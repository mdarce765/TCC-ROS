\documentclass[rascunho,xindy,sublist]{fei}
%\documentclass[rascunho,xindy,sublist]{report}

%\usepackage[utf8]{inputenc}
% \usepackage[table]{xcolor}

%\usepackage{biblatex}

%\usepackage[
%backend=biber,
%style=alphabetic,
%sorting=ynt
%]{biblatex}


\usepackage{pgfplots}
\pgfplotsset{compat=1.7}
\usepackage{pgfplots}
\usetikzlibrary{calc}
\usepackage{amsmath}
\usepackage{subcaption}
\usepackage{booktabs}
\usepackage[utf8]{inputenc}
\usepackage{comment}
\usepackage{acronym} 
\usepackage{glossaries}
\usepackage{hyperref}
\usepackage{url}
\usepackage{xcolor}

% -- Configurações Iniciais

\author{Murilo Darce Borges Silva\\Rodrigo Simões Ruy}

\title{Identificação de diferenças de desempenho entre sistemas robóticos simulados com e sem software conteinerizado, utilizando Simulador Gazebo, ROS 2 e Docker.}
%\subtitulo{subtítulo}

%\cidade{Cidade}
%\instituicao{Instituição de Ensino}

%%%% -- Entradas Listas de Abreviaturas e Simbolos
%%%%%%%%%%%%%%%%%%%%%%%%%%%%%%%%%%%%%%%%%%%%%%%%%%%%%%%%%%%%%%%%%%%%%%%%%%%%%%%%%%%%%%%%%%%%%%%%%%%%%%%%%
%%%% -- Titulos - comentar caso a respectiva lista nao seja utilizada
\newglossaryentry{acro}{name={},description={\nopostdesc},sort=a} %Usado para alinhar a lista de abreviaturas
\newglossaryentry{geral}{name={Geral},description={\nopostdesc},sort=a}
\newglossaryentry{greek}{name={Letras gregas},description={\nopostdesc},sort=b}
\newglossaryentry{sub}{name={Subscritos},description={\nopostdesc},sort=c}

%% -- Abreviaturas
\newacronym[longplural=Computational Aided Design,parent=acro]{cad}{CAD}{Computational Aided Design}
\newacronym[longplural=Centro Universitário da FEI,parent=acro]{fei}{FEI}{Centro Universitário FEI}

%% -- Simbolos
%% -- Latin letters
%\newglossaryentry{}{parent=geral,type=symbols,name={},sort=a,description={}

\newglossaryentry{A}{parent=geral,type=symbols,name={\ensuremath{A}},sort=a,description={exchanger total heat transfer area, $m^2$}}
\newglossaryentry{G}{parent=geral,type=symbols,name={\ensuremath{G}},sort=g,description={exchanger flow-stream mass velocity, $kg/(s m^2)$}}
\newglossaryentry{f}{parent=geral,type=symbols,name={\ensuremath{j}},sort=j,description={friction factor, dimensionless}}

%% -- Greek letters
%\newglossaryentry{}{parent=geral,type=symbols,name={},sort=a,description={}
\newglossaryentry{deltap}{parent=greek,type=symbols,name={\ensuremath{\Delta P}},sort=p,description={pressure drop, $Pa$}}
\newglossaryentry{nu}{parent=greek,type=symbols,name={\ensuremath{\nu}},sort=b,description={specific volume, $m^3/kg$}}
\newglossaryentry{beta}{parent=greek,type=symbols,name={\ensuremath{\beta}},sort=b,description={ratio of free-flow area $A_{ff}$ and frontal area $A_{fr}$ of one side of exchanger, dimensionless}}

%% -- Subscripts
%\newglossaryentry{}{parent=geral,type=symbols,name={},sort=a,description={}
\newglossaryentry{fr}{parent=sub,type=symbols,name={\ensuremath{fr}},sort=fr,description={frontal}}
\newglossaryentry{in}{parent=sub,type=symbols,name={\ensuremath{i}},sort=in,description={inlet}}
\newglossaryentry{out}{parent=sub,type=symbols,name={\ensuremath{o}},sort=out,description={outlet}}
%%%%%%%%%%%%%%%%%%%%%%%%%%%%%%%%%%%%%%%%%%%%%%%%%%%%%%%%%%%%%%%%%%%%%%%%%%%

\makeindex
%\makeglossaries

\begin{document}

\maketitle
\begin{folhaderosto}

%Monografia apresentada ao Curso de Graduação em Ci\^{e}ncia da
%Computa\c{c}\~{a}o do Centro Universitário da FEI, como requisito
%parcial para a obten\c{c}\~{a}o do grau de Bacharel em Ci\^{e}ncia
%da Computa\c{c}\~{a}o.
Trabalho de Conclusão de Curso apresentado ao Centro Universitário FEI, como parte dos requisitos necessários para obtenção do título de Bacharel em Ciência da Computação. Orientado pelo Prof. Dr. Leonardo Anjoletto Ferreira.

\end{folhaderosto}

%\fichacatalografica

\orientador{Prof. Dr. Leonardo Anjoletto Ferreira}
\primeiroexaminador{Prof. Dr. Plinio Thomaz Aquino Junior}
\segundoexaminador{Prof. Dr. Fagner de Assis Moura Pimentel}
\dataaprovacao{01/12/2025}

\begin{folhadeaprovacao}

Trabalho de Conclusão de Curso apresentado ao Centro Universitário FEI, como parte dos requisitos necessários para obtenção do título de Bacharel em Ciência da Computação.

\end{folhadeaprovacao}

%\dedicatoria{A quem eu quero dedicar o texto.}
\begin{agradecimentos}
    Agradecemos ao professor Fagner Pimentel pelo modelo para simulação do labirinto que existe na K4-04, que foi de suma importância para o desenvolvimento do projeto.
\end{agradecimentos}
%\epigrafe{Science never solves a problem without creating ten more}{George Bernard Shaw \nocite{george_bernard}}

\centerline{\bfseries RESUMO}
\label{resumo}
\vspace{8mm}

Conteinerização é uma ferramenta muito útil quando se lida com projetos que precisam de diferentes dependências ou programas que podem ter conflitos entre si, que precisam de uma grande quantidade de configuração inicial, ou que precisam de portabilidade. Isso a torna perfeita para projetos de robótica, mas os impactos do seu uso e suas peculiaridades em situações reais ainda não estão documentadas, o que é justamente oque este projeto propõe fazer. Haverão 2 partes para este projeto, a primeira parte será uma avaliação do desempenho em uma simulação utilizando Gazebo, e a segunda parte será a avaliação do desempenho de um turtlebot real.  

Palavras-chave: Robótica, ROS, Docker, Conteinerização

\centerline{\bfseries ABSTRACT}
\label{abstract}
\vspace{8mm}

Containerization is a very useful tool when dealing with projects that require different dependencies or programs that may conflict with each other, that need a large amount of initial configuration, or that need portability. This project aims to identify and show the performance difference between robots, using Robot Operating System 2 (ROS 2), simulated with and without containerization. To verify this difference, tests will be performed in the simulated environment of the Gazebo Classic software. The results show that the simulation performance is not impacted by the use of Docker, that there is a small increase in memory usage, but without significant variations in processor usage, indicating that simulation using containers is viable during the development process.

\textbf{Keywords:} Robot, Robot Operating System 2 (ROS 2), Gazebo Classic, Docker, Containerization.

% este comando é para ser descomentado apenas na versao final
%\listoffigures

% este comando é para ser descomentado apenas na versao final
%\listoftables

% este comando é para ser descomentado apenas na versao final
%\listofalgorithms

% este comando é para ser descomentado apenas na versao final
%\glsaddall

% este comando é para ser descomentado apenas na versao final
%\printglossaries

\tableofcontents

% este arquivo aqui pode ser descomentado quando forem fazer a introdução
 \chapter{INTRODUÇÃO}
\label{intro}

\begin{comment}
\emph{Questão a ser respondida pela Introdução: O que você fez? E por que você fez isso?}
Na introdução devemos ajudar o leitor a entender o motivo / importância da pesquisa e o que ela está contribuindo para a área. Portanto, indique a motivação (ou motivações) para fazer a pesquisa e o que ela contribuirá para o campo.

Dê também ao leitor uma breve descrição sobre a área de pesquisa de forma geral e ampla e, então, reduza ao foco específico do trabalho, oferecendo informações detalhadas sobre o tópico pesquisado.

Deixe claro quais são os objetivos da pesquisa. 

Pode-se, no final da introdução, apresentar ainda a estrutura do trabalho, indicando como ele está organizado e os tópicos que serão abordados em cada seção subsequente.

Nos últimos anos, a globalização e as transformações tecnológicas vêm redefinindo a estrutura do mercado de trabalho em todo o mundo [1]. Essas mudanças têm afetado profundamente a segurança do emprego e as condições de trabalho dos indivíduos. O objetivo deste trabalho é investigar o impacto da automação e da inteligência artificial na precarização do trabalho, analisando como essas tecnologias têm contribuído para o aumento da informalidade e da desigualdade no mercado de trabalho. Este estudo é relevante em um contexto em que a preocupação com a segurança no emprego e a equidade está no centro das discussões globais.

Nos últimos anos, a Inteligência Artificial (IA) se consolidou como uma das tecnologias mais promissoras e transformadores do nosso tempo [1]. A capacidade das máquinas de aprender e tomar decisões de forma
autônoma tem aplicações em uma variedade de setores, desde a medicina até o transporte [2]. Este estudo tem como objetivo investigar o uso da IA na otimização de sistemas de recomendação em plataformas de streaming de vídeos. Com o crescente volume de conteúdo disponível, tornou-se crucial para as empresas de streaming entender as preferências dos usuários e oferecer recomendações personalizadas. Esta pesquisa é relevante não apenas do ponto de vista da indústria de entretenimento,
mas também do ponto de vista da pesquisa em IA, pois aborda desafios técnicos significativos.
\end{comment}



A conteinerização é uma ferramenta poderosa no campo de desenvolvimento e implementação, disponibilizando certa camada de isolamento entre componentes de um projeto, assegurando que estes não irão conflitar, seja por funções internas ou dependências de versões diferentes sendo utilizadas. No campo da robótica, conteinerização é vista como uma técnica para facilitar o desenvolvimento, portabilidade e consistência em projetos de robótica, mas não foram feitas pesquisas detalhando a integração destes projetos com Docker e seus efeitos no desempenho de um robô físico. A proposta do projeto é justamente esta: integrar ROS 2 e Docker, explicando os passos utilizados e comparando o desempenho com e sem conteinerização, sendo dividido em 2 partes: em um simulador Gazebo, e em um robô Turtlebot real.
\begin{comment}
    \begin{itemize}
    \item Contexto
    \begin{enumerate}
        \item Como esse tópico se encaixa no contexto da área de pesquisa?\\
        R: Se encaixa na pesquisa pois estamos utilizando técnicas da área da ciência da computação para implementar e avaliar o desempenho de ROS + Docker.
        \item Qual a relevância do tópico escolhido para a área de estudo?\\
        R: O trabalho é relevante pois pode ajudar futuros projetos utilizando ROS+Docker, ao evidenciar certas falhas e/ou perdas de desempenho relacionado com essa integração
        \item Quais eventos históricos e/ou recentes que contextualizam esse tópico?\\
        R: O crescimento do uso de containers em projetos com grande quantidade de complexidade/dependências [colocar artigo aqui]
    \end{enumerate}
    \item Problema de pesquisa
    \begin{enumerate}
        \item Qual a questão central que seu TCC se propõe a abordar?\\
        R: Diferenças de desempenho relacionada à integração Docker, além de possível incompatibilidades/peculiariades relacionadas.
        \item Por que esse problema é significativo ou merece investigação?\\
        R: Pois sem saber destas diferenças, é difícil fazer a decisão de qual partes do projeto podem ser integradas a Docker, e qual os possíveis riscos.
        \item Existem lacunas na literatura existente relacionada a esse problema?\\
        R: Sim, já foram feitos vários estudos relacionados ao desempenho do ROS+Docker em simulações, ou sua inicialização [colocar artigo aqui], mas não no robô físico.
    \end{enumerate}
    \item Objetivos
    \begin{enumerate}
        \item Quais são os principais objetivos do seu TCC?\\
        R: Encontrar e avaliar diferenças na performance e compatibilidade de projetos ROS e ROS+Docker
        \item Como seus objetivos estão relacionados ao problema de pesquisa?\\
        R: Ter estes dados irá facilitar a avaliação e desenvolvimento de futuros projetos ROS+Docker.
        \item O que você espera alcançar com esta pesquisa?\\
        R: A obtenção de dados relevantes à decisão de integrar Docker ao projeto, e quais partes deste podem ganhar com esta integração.
    \end{enumerate}
    \item Justificativa
    \begin{enumerate}
        \item Por que é importante abordar esse problema de pesquisa?\\
        R: 
        \item Quais são as implicações práticas e teóricas da sua pesquisa?\\
        R: 
        \item Como sua pesquisa contribuirá para o conhecimento existente na área?\\
        R: 
    \end{enumerate}
    \item Hipótese
    \begin{enumerate}
        \item Qual é a suposição sendo feita com base na revisão da literatura?\\
        R: Que existem diferenças que não se mostram em simulações.
        \item Como sua suposição se relaciona ao problema de pesquisa?\\
        R: 
        \item Como você pretende testar ou validar essa hipótese?\\
        R: Utilizando métricas como tempo e aproveitamento para medir o desempenho, e uma medição binária para compatibilidades
    \end{enumerate}
\end{itemize}
\end{comment}


\section{OBJETIVO}
Documentar e avaliar o desempenho do robô simulado e do robô real quando comparados à sua implementação com e sem conteinerização.

\begin{comment}
\section{ESTRUTURA DO TRABALHO}

O restante deste trabalho é dividido da seguinte maneira: na Seção 2, serão
apresentados todos os conceitos utilizados e relacionados ao tema abordado, para que o leitor possa entender com clareza as técnicas que estão sendo tratadas no trabalho e compreender os termos que serão descritos posteriormente.

Na Seção 3, os trabalhos relacionados disponíveis na literatura, com o objetivo de apresentar o cenário atual de pesquisa da área.
 
A Seção 4 detalhará a metodologia que será utilizada para o desenvolvimento deste trabalho, demonstrando as técnicas que serão utilizadas e os passos a serem realizados para atingir o objetivo final.

O Seção 5 irá expor o que os autores deste trabalho esperam ao longo do desenvolvimento e após a implementação da metodologia proposta.

\end{comment}



% este arquivo aqui pode ser descomentado quando forem fazer os conceitos
\chapter{CONCEITOS FUNDAMENTAIS} 
\label{conceitos}

Este capítulo aborda os conceitos teóricos e as ferramentas utilizadas no desenvolvimento deste trabalho, com o intuito de descrever e relacionar estes conceitos, para haver entendimento nas etapas posteriores abordadas. O principal conceito a ser abordado é a 
Containerização, uma forma de virtualização feita para ser mais rápida e flexível que a emulação, é um processo de implantação que consegue ser executado em diversos dispositivos e sistemas operacionais. Isso ocorre pelo fato de que um contêiner consegue armazenar os arquivos e bibliotecas para ser executado, permitindo a um usuário executar uma aplicação de outro sistema operacional no sistema operacional que o mesmo possua. Além disso, o contêiner permite que falhas ocorram sem afetar outros processos que não estão agrupados no mesmo. \cite{Wen2023}. A containerização será feita usando o 
Docker é uma plataforma utilizada para desenvolvimento, envio e funcionamento de aplicações de maneira separada da infraestrutura por conta da conteinerização. Por conta deste fator, o usuário consegue gerir as aplicações da mesma maneira que gera sua infraestrutura. Outro fator importante é que o Docker permite que as aplicações desenvolvidas sejam testadas e executadas com menos atraso do que a maneira convencional. Contêineres são bons para fluxos de integrações e entregas de trabalho contínuas. \cite{dck2025}
Para realizar os testes no robô simulado, será utilizado o ROS2 (Robot Operating System 2), que é um meta-sistema operacional. 
Um meta-sistema operacional é um sistema operacional para robôs, o mesmo realiza funções similares a outros sistemas operacionais, com exceção de controles de CPU, pois o mesmo executa processos robóticos, fluxos de dados, entre outros. Possui bibliotecas e ferramentas que executam códigos em múltiplos computadores. Este conceito é utilizado pelo ROS 2, sendo o método mais utilizado nos robôs atuais, sendo uma camada acima de um sistema operacional real, que oferece abstrações e serviços para os sistemas robóticos.
O ROS 2 é justamente um meta-sistema operacional de código aberto utilizado para auxiliar a desenvolver aplicações para robôs. O mesmo possui serviços que outros sistemas operacionais normalmente possuem, mas com o foco maior para a área da robótica, facilitando comunicação entre processos, funções que se comunicam com as demais e entre muitos outros. Para o desenvolvimento do projeto, será utilizado o ROS 2, que mantém o conceito modular e distribuído, mas possui melhorias e mais funcionalidades que o ROS original (Figura \ref{fig:ROS25}) \cite{ros2025}. Para conectar as aplicações, foram utilizados DDS, que são protocolos Middleware. Os
Middlewares são uma camada de software que conecta as aplicações a um sistema operacional, permitindo uma comunicação e compartilhamento de dados mais simples entre os componentes de um sistema. Esta facilidade permite que os desenvolvedores foquem no desenvolvimento das principais funções de uma aplicação, pois a comunicação entre a aplicação e o sistema operacional está sendo feita pelo middleware. O
DDS é um protocolo middleware e uma API para conexão centrada em dados, este protocolo integra os componentes de um sistema que muitas aplicações precisam, como arquitetura escalável, confiabilidade e prover conectividade de dados de baixa latência. Este protocolo foi criado pela OMG. Para este projeto, foram utilizados dois tipos de DDS, o primeiro é o
Fast DDS uma implementação de DDS feita em C++, possui uma biblioteca que oferece uma API e protocolo de comunicação que disponibiliza um modelo Publisher-Subscriber centrado em dados (DCPS). Este modelo visa ser eficiente e confiável para distribuir as informações para o sistema em tempo real. O segundo DDS utilizado é o
Cyclone DDS, que é uma implementação de DDS com alto desempenho, permite que os desenvolvedores que o utilizam possam criar "gêmeos" digitais das entidades de seus sistemas, permitindo compartilhar estados, eventos, fluxos de dados e mensagens pela rede em tempo real, visa ser rápido, consistente e seguro.

Os testes simulados serão executados em modelos baseados na arena presente na sala K4-04 do Centro Universitário FEi, para isso, será utilizado o software
Gazebo Simulator Classic que É um software usado para desenvolver simulações, possui diversos projetos de código aberto para que os interessados possam utilizar e desenvolver suas próprias simulações. Neste software estão presentes também diversos modelos, tanto como objetos como também robôs.(Figura \ref{fig:gzb25}) \cite{gzb2025}. Para realizar estes testes, será utilziado o modelo virtual do robô 
TurtleBot3 Burger que É um robô customizável de preço acessível ao público baseado no modelo ROS para ser utilizado como um material educativo, de pesquisas, entretenimento pessoal e etc, é um robô que foi desenvolvido com o intuito de ser barato, por conta disto, o mesmo não possui uma grande funcionalidade ou qualidade, mas o mesmo compensa na relação da quantidade de aplicações que o mesmo consegue realizar.(Figura \ref{fig:tbt3b25}) \cite{turtlebot3_manual}.

\begin{figure}[htb]
    \centering
    \includegraphics[width=0.3\linewidth]{Figures/ROS.png}
    \caption{Exemplo do ROS 2, utilizando turtlesim, ros e rqt \cite{rosrqt}}
    \label{fig:ROS25}
\end{figure}

\newpage
\begin{figure}[htb]
    \centering
    \includegraphics[width=0.7\linewidth]{Figures/Gazebo.png}
    \caption{Exemplo do Gazebo Simulator Classic com a arena Presente na K4-04 \cite{gzb2025}}
    \label{fig:gzb25}
\end{figure}

\begin{figure}[htb]
    \centering
    \includegraphics[width=0.4\linewidth]{Figures/TurtleBot3.jpg}
    \caption{TurtleBot3 Burger \cite{tbt2025}}
    \label{fig:tbt3b25}
\end{figure}

% este arquivo aqui pode ser descomentado quando forem fazer os trabalhos relacionados
\chapter{TRABALHOS RELACIONADOS}
\label{trabs}

\novo{Este capítulo apresenta trabalhos que tratam de conteinerização de forma próxima a proposta deste trabalho, ou seja, tanto sobre a comparação de desempenho no uso de contêiner quanto do uso com o ROS para projetos de robótica.}

\anot{DESENVOLVER, COLOCAR METODOLOGIA DOS PROJETOS}
\section{Bare-Metal vs. Hypervisors and Containers: Performance Evaluation of Virtualization Technologies for Software-Defined Vehicles}

% \begin{itemize}
%     \item Materiais:
%     \item Métodos:
%     \item Métricas:
% \end{itemize}

% \textbf{Resumo}

% \textbf{Contribuições}

% \textbf{Limitações}

\ante{O projeto de Wen et. al. (2023), auxilia com relação ao entendimento da conteinerização em sistemas embarcados e também com relação ao seu desempenho.No artigo, é detalhada}
\novo{O artigo de \citeonline{Wen2023} detalha }
a utilização de diferentes formas de conteinerização e seu efeito no desempenho em diferentes tipos de hardware. Foram realizados testes gerais que envolviam CPU, memória, rede e disco, em três ambientes diferentes: 
\ante{máquina virtual, contêiner e um contêiner dentro de uma máquina virtual.}
\novo{Integrado (utilizando uma Raspberry Pi 4 Modelo B), Computador desktop (Dell Optiplex 7040 PC) e uma estação de trabalho customizada de alto desempenho}
\novo{Para realizar estes testes, foram utilizados Whetstone (versão 1.2), Dhrystone (versão 2.2a) e Kcbench (0.9.5), ferramentas de medição de desempenho, para realizar os testes de CPU. Os testes de memória foram medidos pelo software RAMspeed (versão 3.5.0). Foi utilizado o iperf3 (versão 3.9) para medir os testes de rede, enquanto para analisar o desempenho do disco, utilizou-se o Dbench (versão 4.00), Bonnie++ (versão 2.00) e o Sysbench (versão 1.0.20). Foram utilizados e analisados diversos contêineres engine, sendo estes o Docker (versão 20.10.17), KVM (v1.25.3), Podman (versão 3.4.4) e Systemd-Nspawn (versão 249.11). Para realizar os testes em diversos contêineres, foi utilizada uma versão leve do Kubernetes, o k3s (versão V1.25.3). Foi utilizado o Autoware, um framework para realizar simulações de condução autônoma, para verificar a inicialização dos ambientes virtualizados e contêinerizados.}

\novo{Os testes de CPU consistiam em avaliar o desempenho no bare-metal (servidor físico que é de uso exclusivo para apenas um cliente, sem a camada de virtualização que fica entre o hardware e o sistema operacional), contêineres (Docker e Podman) e em virtualizações (KVM). Para isso, utilizou-se o Whetstone para realizar 50 milhões de cálculos de ponto flutuante para medir a eficiência da CPU (a métrica usada foi o mflops, que são as milhões de operações realizadas por segundo), o Dhrystone para executar um código de cálculo simples continuamente, para analisar as operações comuns da computação (onde analisou as interações realizadas por segundo como métrica) e o Kcbench para compilar o código do kernel do Linux e medir quanto tempo para a CPU completar a tarefa (utilizou-se a métrica de Kernel/hora).
Ao medir a memória, foi utilizado o RAMspeed, que analisa a velocidade da escrita e leitura na memória e quanto tempo se leva para realizar essas operações, repetindo este processo várias vezes nas três plataformas (a métrica para este teste foi a velocidade de escrita e leitura por segundo).
Para os testes de rede, utilizaram o iperf3, que mede a largura de banda de envio e recebimento de dados por meio de transmissões TCP e UDP (o Throughput, sendo a taxa de transferência, e a latência foram as métricas deste teste).
Por último, foi analisado o desempenho do disco, que foi medido pelo Dbench, que simula uma aplicação de operações de Input/Output em sistemas de arquivos, e o Bonnie++, que testa arquivos realizando operações de criação, leitura, gravação e exclusão de arquivos (as métricas usadas foram throughput de leitura e gravação, medido em megabytes por segundo, e o tempo para criar e excluir arquivos)}.
Com isso, foi concluído que as máquinas virtuais e os contêineres possuem um desempenho semelhante ao bare-metal, onde entre a CPU, rede e memória, sofria uma perda de no máximo 5\% enquanto no disco a diferença era de até 35\%. Foi observado que o Docker e a KVM (máquina virtual baseada no Kernel) foram 5 a 10\% mais lentos que o bare-metal, com o Docker sendo mais lento ainda na primeira inicialização, mas levando a concluir que conteinerização e virtualização podem ser utilizados em aplicações para automóveis. \novo{Este trabalho auxilia no entendimento da aplicação e avaliação de ambientes conteinerizados físicos e virtualizados. Para o desenvolvimento deste trabalho, serão utilizadas as análises realizadas com relação ao Docker e suas aplicações}.

\begin{comment}
- CPU: Whetstone, Dhrystone, Kcbench
- Memória: RAMspeed
- Rede: iPerf3
- Disco: Dbench, Bonnie++, Sysbench
Além disso, o framework Autoware (utilizado para simulação de condução autônoma) foi testado em ambientes virtualizados e contêinerizados para medir o impacto na inicialização do sistema, que é crucial para aplicações automotivas.
\end{comment}
\section{Docker Performance Evaluation across Operating Systems}

O
\ante{trabalho}
\novo{artigo}
de \citeonline{SMKD2024} auxilia no entendimento dos conceitos de avaliação do Docker com relação a outros sistemas operacionais.
\novo{Este estudo realiza testes onde o Docker é executado em diversos sistemas operacionais instalados em um MacBook Pro 13, com uma CPU da Intel i5-8257u @ 1.40GHz, 16GB LPDDR3 2133 Mhz de RAM e  256GB NVME SSD de armazenamento, com a versão 4.20 do Docker Engine e configurado para utilizar 8 CPUs lógicos, 16GB de RAM e 1GB de memória swap}.
Para verificar a diferença entre os sistemas operacionais, 
\ante{foram utilizados sistemas operacionais recém-instalados que eram macOS Ventura, Ubuntu 22.04 e Windows 10 rodando em um MacBook Pro 13}
\novo{foram utilizados os sistemas operaiocnais MacOS Ventura 13.5.1, Ubuntu 22.04-6.4.8-t2-jammy e Windows 10 22H2 recem-instalados}.
\ante{Os testes consistiam em estressar o Docker com relação à CPU, rede e na resiliência do mesmo.}
\novo{Neste projeto, o objetivo foi estressar o Docker para medir o desempenho da CPU usando um programa escrito em linguagem C para realizar o cálculo de Pi pela fórmula de Leibniz, sendo a métrica escolhida o tempo de execução do programa. Outro teste foi o teste do Sysbench, em que eram realizados cálculos dos números primos para verificar o desempenho do processador, com as métricas sendo a quantidade de operações por segundo.}

\novo{Além disso, foram feitas diversas leituras e gravações aleatórias e sequenciais no disco rígido para medir o tempo de resposta e taxa de transferência das operações (teste de Input/Output). As métricas foram os testes de leitura e escrita.
O teste do Iperf3 foi realizado a comunicação entre contêineres usando o protocolo TCP, depois utilizando o UDP, e por último a comunicação entre o host e o contêiner usando novamente o TCP. Este teste tinha como foco verificar o desempenho da rede, as métricas usadas foram Throughput (taxa de transferência) em TCP e UDP e a perda de pacotes.
Em seguida, foi realizado o teste utilizando o 7Zip, em que era medida a velocidade de compactação e descompactação dos arquivos, e verificar o comportamento e desempenho do Docker e do sistema operacional.
Foi realizado o teste de verificação de desempenho de um banco de dados no Docker, foi utilizado o Pgbench, onde era medida a taxa de transações por segundo (TPS) e verificado como o desempenho variava.
Por último, foi feito o teste de Apache slowhttp attack, em que foram abertas diversas conexões HTTP lentas e, com isso, testada a capacidade e resistência ao sofrer um ataque de negação de serviço (Denial of Service - DoS) e verificar qual sistema acabou sofrendo mais. Foi medido o tempo de resposta durante o ataque DoS.}
Após realizar os testes, foi observado que o sistema operacional possui uma importante influência sobre o desempenho presente no conteiner. Alguns possuíam benefícios em relação a outros em uma determinada categoria. O macOS se destacou com relação aos dados obtidos nas configurações utilizadas nos sistemas docker, não sofrendo grandes perdas de desempenho, se mostrando extremamente versátil,
o Linux se mostrou mais eficiente quanto às aplicações que raramente utilizam escrita no disco, se mostrando uma escolha melhor que o MacOS com relação a bancos de dados em memória, cache e entre outros, pois por não necessitarem de tanta escrita, essas aplicações se beneficiam mais quando executadas em um conteiner com Linux,
o Windows acabou não se beneficiando tanto quanto os outros, a não ser pela taxa de transferência de rede entre os contêineres. Assim como o Linux, o Windows possui problemas com a velocidade em que a escrita é feita e com isso.
\ante{Este artigo auxilia no entendimento com relação aos tipos de testes que podem ser realizados para analisar o desempenho do Docker, auxiliando de uma maneira que possa ser um pontapé inicial para o desenvolvimento dos testes com o Docker, e como medir o desempenho de um contêiner}
\novo{Este trabalho auxilia com o entendimento sobre o desempenho do Docker em diferentes sistemas operacionais, mostrando qual opção mais se aplica a este trabalho, que no caso deste trabalho, foi o Linux, por ser mais acessível que o macOS e por possuir um desempenho superior ao do Windows, além de mostrar tipos de testes que podem ser realizados para verificar o desempenho e quais métricas utilizar}.

\section{DA Containerized Microservice Architecture for a ROS 2 Autonomous Driving Software: An End-to-End Latency Evaluation}

\ante{Conforme Wen et. al. (2024), o artigo} 
\novo{O trabalho de \citeonline{Wen2024}} 
aborda a arquitetura baseada em microserviços para sistemas automotivos 
\novo{autônomos usando ROS 2}. 
Cada serviço foi 
\ante{realizado}
\novo{executado} em contêineres separados, pois os testes realizados identificaram que este método é viável e acaba por melhorar a latência existente em sistemas operacionais Linux sem contêineres, que obteve uma latência de 5 a 8\% end-to-end, além de reduzir a latência máxima, o que mostra a vantagem no uso de conteinerização para os sistemas automotivos em tempo real.
\novo{Neste projeto foram utilizados duas plataformas de computação distintas, uma com arquitetura x86 (InoNet Mayflower-B17) e outra aarch64 (Armv8 ADLINK AVA COM-HPC) com ambos utilizando o sistema operacional Ubuntu 22.04.3 LTS Jammy Jellyfish, GPU NVIDIA RTX A6000 48 GB e Kernel 6.2.0-34-generic, foi utilizado o ROS 2 Humble Hawksbill com o middleware Eclipse CycloneDDS, Docker engine (versão 24.0.5), k3s (versão v1.27.3+k3s1)}.
Foi concluído que o ROS 2, utilizado para avaliar a arquitetura de microserviços para uma aplicação real de direção autônoma, foi de extrema importância por conta de sua arquitetura distribuída que é baseada em nós e possui comunicação DDS, o que levou ao isolamento das funções do veículo e facilitou a migração para contêineres. Na conteinerização, o ROS 2 perde um pouco de seu desempenho, mas em aplicações complexas como o Autoware, a mesma acaba por
\ante{melhorar}
\novo{ter um desempenho melhor},
reduzindo o uso de CPU e memória. 
\ante{Este artigo auxilia no entendimento do uso da conteinerização com relação ao ROS 2, seus problemas e seus acertos}
\novo{Este artigo demostrou que o uso de containeres para o tipo de problema proposto neste trabalho é viável e que pode trazer benefícios em relação a execução em bare-metal}.

% este arquivo aqui pode ser descomentado quando forem fazer a metodologia
\chapter{METODOLOGIA}
\label{metodologia}

A metodologia (Observada na \ref{fig:Fluxograma}) é dividida em 2 partes. Na primeira parte, será utilizada uma simulação no gazebo simulator. A simulação possuirá um ambiente para testar o desempenho de um robô turtlebot. Ocorrerão dois testes, no primeiro teste o robô utilizará apenas ROS 2 enquanto no segundo teste o robô utilizará ROS 2 e Docker. Após a análise[EXPLICAR COMO É FEITA A ANÁLISE] do desempenho nos testes, os dados serão armazenados, analisados e finalmente comparados.
Na segunda parte, ao invés de uma simulação, será utilizado um robô turtlebot real. Os testes são semelhantes aos aplicados na primeira parte, onde o primeiro teste será realizado apenas com ROS 2 enquanto o segundo será realizado com ROS 2 e Docker juntos, para que assim os dados de desempenho possam ser coletados, analisados e então comparados.
As arenas que serão utilizadas na simulação serão baseadas em possíveis arenas que serão montadas na FEI na sala k404. A mesma possui uma arena que pode ser montada e um robô TurtleBot 3 Burger para a realização dos testes, assim será desenvolvida uma simulação com uma arena baseada na arena existente na instituição. A arena possui um caminho, mas o mesmo pode [COLOCAR UMA IMAGEM DA ARENA DA K404] ser alterado com algumas placas que funcionam como paredes, essas paredes permitem a criação de um caminho diferente para o robô podendo ser feito um labirinto, assim seria desenvolvida uma simulação com a mesma ideia para que se pudesse obter dados com uma relação maior. Com relação à parte física, seriam utilizados tanto o robô quanto a arena física para que pudessem ser executados os testes necessários para se adquirir os dados para então analisá-los e compararmos.

\begin{figure}[htb]
    \centering
    \includegraphics[width=0.9\linewidth]{Figures/FluxogramaTCC.png}
    \caption{Fluxograma do projeto[NECESSITA SER REFERENCIADO]}
    \label{fig:Fluxograma}
\end{figure}

% este arquivo aqui pode ser descomentado quando forem fazer a proposta experimental
\chapter{EXPERIMENTOS}
\label{experimentos}

\novo{Este capítulo apresenta os experimentos realizados que utilizam os conceitos apresentados no capítulo anterior.}

\novo{\section{EXPERIMENTOS EM SIMULAÇÃO}}
Os experimentos usam um modelo digital da arena física presente no laboratório da sala K4-04 da instituição 
\novo{e usam um robô TurtleBot3 Burger simulado disponível no ROS e Gazebo}. O motivo de esta
%\ante{arena}
\novo{combinação} ter sido escolhida se deu por conta da comparação que
%\ante{futuramente} pode ser realizada entre os testes virtuais e reais.

Foram
%\ante{feitos cinco}
\novo{propostos sete} experimentos com resultados satisfatórios, e um que acabou não fornecendo os resultados devidos por conta de instabilidades.
\novo{
Todos os testes foram feitos em um computador com as seguintes especificações:
Linux Jammy 22.04 com um processador AMD Ryzen 5 1600X (12 Cores, 3.6GHz), 16 GB RAM DDR4.
Todas as ferramentas e configurações destas estão disponíveis em um playbook, no repositório.
A imagem Docker utilizada no projeto se baseia na imagem usada para ROS 2 Humble, presente no repositório oficial da robotis\footnote{https://github.com/ROBOTIS-GIT/turtlebot3/tree/main/docker/humble}, modificada para ter as configurações e arquivos necessários.
}

%\ante{O primeiro deles (Setup)teve testes iniciais para confirmar o comportamento do ROS 2 dentro de um container Docker.No segundo experimento (Base), usaram-se simulações utilizando o mapa base sem patrulheiros. O terceiro experimento (Patrulheiros) usou simulações utilizando o mapa base com patrulheiros. O quarto experimento (Base + Docker) usou simulações utilizando o mapa base sem patrulheiros, com a stack nav2 e rviz dentro de um container Docker. No quinto experimento (Patrulheiros + Docker), usaram-se simulações utilizando o mapa base com patrulheiros, com a stack nav2 e rviz dentro de um container Docker.}
Foram também realizados experimentos usando arenas com labirintos para testar a navegação do robô,
%\ante{mas a navegação acabou tendo problemas em experimentos com e sem Docker, dando a indicar que é um problema com o mapa, ou com as bibliotecas ROS 2 utilizadas, e não com a conteinerização.} 

\novo{Os sete experimentos propostos foram:
\begin{enumerate}
    \item Setup: testes iniciais para confirmar o comportamento do ROS 2 dentro de um contêiner Docker e a viabilidade do projeto proposto.
    \item Arena: navegação do robô de um lado a outro na arena da sala K4-04 sem nenhum obstáculo, conforme o caminho exibido na Figura~\ref{fig:AB}.
    \item Patrulheiros: usa o mesmo mapa do experimento anterior, porém 3 robôs TurtleBot3 Burger foram adicionados para realizar um percurso pré-definido (Figura~\ref{fig:AP}) com o objetivo de atrapalhar a movimentação do robô que faz a navegação na arena. Os robôs patrulheiros faziam um trajeto em linha reta onde saiam e retornavam aos seus pontos de origem.
    \item Arena com Docker: Este experimento é idêntico ao experimento Arena, porém o robô executa a parte do ROS de dentro de um contêiner Docker.
    \item Patrulheiros com Docker: Assim como o anterior, este experimento repete o que foi realizado no experimento Patrulheiros mas com o robô usando o ROS em um contêiner.
    \item Labirintos: dois mapas com paredes (Figuras~\ref{fig:AL1} e~\ref{fig:AL2}) foram criados para verificar se ocorre alguma alteração no desempenho do robô em relação a navegação. O uso destas paredes se devem ao fato da arena física presente na K4-04 e os mapas desenhados respeitam o espaço da arena, a quantidade de placas disponíveis e as possíveis posições que estas podem ser adicionadas na arena.
    \item Labirintos com Docker: assim com os anteriores, este experimento replica o que foi proposto no Labirintos, mas com o robô executando o ROS de dentro do contêiner.
\end{enumerate}
}
\begin{figure}[ht!]
\caption{Modelos de arena usadas para os experimentos simulados.}
\centering
\begin{subfigure}[t]{0.45\textwidth}
    \caption{Arena base com robô TurtleBot3 Burger. A linha verde representa o trajeto que deve ser realizado pelo robô.}
    \includegraphics[width=\textwidth]{Figures/Arena/ArenaBase.png}
    \label{fig:AB}
\end{subfigure}
\hfill
\begin{subfigure}[t]{0.45\textwidth}
    \caption{Arena Patrulheiros com 4 robôs TurtleBot3 Burger. As retas em vermelho são os trajetos realizados pelos patrulheiros, enquanto a curva em verde, representa o trajeto do robô.}
    \includegraphics[width=\textwidth]{Figures/Arena/ArenaPatrulheiros.png}
    \label{fig:AP}
\end{subfigure}
\begin{subfigure}[b]{0.45\textwidth}
    \caption{Labirinto 1. Labirinto feito adicionando paredes à arena da K4-04.}
    \includegraphics[width=\textwidth]{Figures/Arena/ArenaLabirinto1.png}
    \label{fig:AL1}
\end{subfigure}
\hfill
\begin{subfigure}[b]{0.45\textwidth}
    \caption{Labirinto 2. Segunda arena montada para testes com obstáculos.}
    \includegraphics[width=\textwidth]{Figures/Arena/ArenaLabirinto2.png}
    \label{fig:AL2}
\end{subfigure}
\caption*{Fonte: Autores}
\end{figure}

\novo{Para todos os experimentos as métricas de uso de CPU, RAM e rede foram medidas usando o nmon e armazenadas em arquivos para que os resultados fossem analisados após a execução de todos os testes. Cada experimento foi executado pelo menos 30 vezes para garantir que a amostragem dos resultados é estatisticamente relevante.}

\novo{\section{Experimentos no robô real}}

\novo{Os experimentos utilizaram somente um robô TurtleBot3 Burger que utilizava a Raspberry Pi 4 Model B Rev 1.5, com 4 núcleos, utilizando um processador Quad core Cortex-A72 (ARM v8) 64-bit SoC 1.8GHz com 2GB de memória RAM. O robô utilizava também Ubuntu 22.04.5LTS, Docker 28.4.0 (esta imagem foi baixada do repositório do Ubuntu) e ROS 2 Humble Hawksbill. Os testes foram realizados somente na arena base (sem obstáculos) na sala K4-04 (Figura \ref{fig:AB}). Estes testes consistiam em repetir o teste de navegação simulada, mas na arena física. Neste teste, o robô precisa sair de uma ponta e se locomover até a outra. O teste iniciava quando o Nmon era acionado por meio de um script armazenando os dados da CPU, memória RAM e tráfego de rede, e então, com outro script, o robô inicializava o xterm que inicializava o rviz. Com isso, o robô começava a se locomover. Quando o robô finalizava o trajeto, eram executados os scripts para encerrar o Nmon e o xterm, com isso o robô era reposicionado e depois o teste era repetido mais 29 vezes, este teste foi executado tanto sem quanto com o Docker. Foram executados no total 60 testes, divididos entre sem e com Docker. Após os dados serem obtidos, os mesmos eram analisados no software NMONVisualizer e então convertidos para “.csv” para serem utilizados em gráficos}.

\section{CONTEÚDO REPOSITÓRIO}
\novo{
O repositório GitHub\footnote{Disponível em https://github.com/joca2511/TCC\_Docker} criado para este projeto possui várias ferramentas que podem ser utilizadas para facilitar a reprodução dos testes executados.
O arquivo \texttt{README.md} contém um overview de como instalar o projeto e utilizar os scripts presentes no projeto.
Foi criado um playbook ansible (arquivo \texttt{playbook.yaml}) para facilitar a configuração e instalação do Docker, ROS 2 Humble, Gazebo Classic e suas dependências.
Foram também criados vários scripts para shell (arquivos com extensão \texttt{.sh}) para facilitar a reprodução dos testes, para garantir que os comandos corretos serão executados na sequência correta, sem necessidade de intervenção do usuário, onde os arquivos \texttt{inicioRapido*.sh} possuem os comandos utilizados para cada um dos testes feitos para este projeto.
Os arquivos com extensão \texttt{.sh} na pasta \texttt{/scripts} possuem funcionalidades genéricas criadas para os testes, como a inicialização do nmon (script \texttt{iniciarNmon.sh}), mover o robô ao inicializar rviz e mandar uma mensagem de goal (\texttt{moverMain.sh}), entre outras.
No pacote ROS 2 criado (localizado na pasta /tcc/tcc) possui o arquivo \texttt{turtlebot3\_absolute\_move\_Arena.py} que é uma versão modificada do arquivo \texttt{turtlebot3\_absolute\_move.py}, utilizado na movimentação dos robôs patrulheiros.
Esta modificação faz com que os robôs inicializados entrem em um loop infinito de movimentação entre 2 pontos especificados, lógica utilizada no script \texttt{rotasRobos.sh}.
Há também uma versão modificada de \texttt{multi\_robot\_Arena.launch.py} com as posições iniciais dos robôs patrulheiros e do arquivo \texttt{turtlebot3\_world.launch.py} modificado em \texttt{turtlebot3\_Arena.launch.py}, que permite carregar o robô em uma posição fixa em qualquer mapa, contanto que o nome seja especificado e o arquivo \texttt{.world} presente em /tcc/worlds.
}

%\ante{\section{TESTES ARENA} Foram feitos testes para adquirir os dados de desempenho das simulações sem Docker, utilizando a arena base sem patrulheiros (Figura \ref{fig:AB}). Estes testes consistiam em fazer o robô percorrer um trajeto de uma ponta até a outra da arena. Durante a execução, foi utilizado o software Nmon para verificar o desempenho de CPU, memória RAM e rede, e assim armazenar os dados. Os testes com a arena base + Docker, o robô realizou o mesmo trajeto que o teste anterior de seguir de uma ponta da arena até a outra. Os testes foram realizados enquanto o software Nmon verificava o desempenho do sistema.}

%\ante{\section{TESTES PATRULHEIROS}}
Neste teste foram posicionados robôs patrulheiros que percorriam um
%\ante{simples trajeto} 
\novo{trajeto de linha reta onde saiam e retornavam as seus pontos de origem (Figura \ref{fig:AP})}
para servir de obstáculo para o robô que deveria percorrer o mesmo trajeto de antes. Novamente foi utilizada a arena base, e sem o uso do Docker. Foram feitos testes para adquirir os dados de desempenho de simulações com Docker, utilizando a arena base com patrulheiros.


% este arquivo aqui pode ser descomentado apenas em TCC2
\chapter{CONCLUSÃO}
\label{conclusao}

A partir dos resultados obtidos por meio dos experimentos realizados, \textcolor{red}{pode ser constatado que o uso do Docker gera um impacto no desempenho do robô simulado.} \textcolor{green}{MAIS CERTEZA NA AFIRMAÇÃO} Esse impacto pode ser medido e observado por meio do consumo do desempenho da CPU, memória RAM e tráfego de rede, obtidos por meio da ferramenta Nmon, que mediu os testes realizados no simulador gazebo, onde o robô realizou trajetos com e sem o uso do Docker. Foi observado que a diferença no desempenho foi mínima, mostrando que o uso do Docker não compromete o funcionamento da simulação. Com isso, foi concluído que o Docker pode ser utilizado, mas pode ter alguns problemas, pois não foram realizados testes com tarefas críticas ou com pouco recurso disponível, para realizar testes simulados utilizando o ROS 2, o que acaba por gerar praticidade e portabilidade para o usuário que não precisa se preocupar com prejuízos significativos. 

\begin{comment}
    A partir dos resultados obtidos nos experimentos realizados, constatou-se que o uso do Docker realmente gera impacto no desempenho de um robô simulado. Esse impacto pôde ser medido por meio da análise dos consumos de CPU, memória RAM e tráfego de rede, monitorados de forma detalhada utilizando ferramentas como o nmon, além dos próprios utilitários embarcados no Docker. Apesar disso, observou-se que a degradação de desempenho é mínima e não compromete o funcionamento geral do sistema. Assim, conclui-se que o Docker pode ser empregado de forma segura e eficiente em aplicações robóticas baseadas em ROS/ROS2, oferecendo praticidade e portabilidade sem introduzir prejuízos significativos ao usuário.
\end{comment}
\section{TRABALHOS FUTUROS}
Devido a problemas com relação ao uso da sala K4-04 e seus equipamentos e à falta de tempo, os testes em um robô real não foram realizados, isso acabou por alterar o escopo do projeto, removendo o componente real das avaliações, sendo decidido que os testes e avaliações do robô real seriam realizados como um trabalho futuro.

Como trabalho futuro, ficaria a avaliação e estudo da integração em um robô real, utilizando as ferramentas, arquivos e metodologias criadas por este projeto para facilitar a obtenção de dados, como a Dockerfile customizada, os scripts para fácil utilização e organização de vários comandos.

\textcolor{red}{Onde esta o trabalho de IHC com trabalho futuro?
O texto está muito raso. Falta aprofundamento. Provavelmente o grupo fez mais do que essa descrito.
Minha sugestão é que o grupo realize os testes no ambiente real e apresentem futuramente um texto mais completo.
Banca ROS2 +Docker
- O TurtleSim foi utilizado?
- Slide 5 e 6 - Podem colocar as imagens dos trabalhos relacionados no texto.
- O que dos trabalhos relacionados foi utilizado no trabalho?
- Ficou claro na apresentação o que eram os patrulheiro 
- Imagens nos slides dos experimentos devem estar no texto
- Gostaria de mais detalhes dos testes de navegação. Não faz sentido ele se perder. O ambiente estava mapeado?
- A explicação na apresentação está melhoro que do texto}


%\bibliographystyle{plain}
\bibliography{tcc-referencias}
%\addbibresource{tcc-nome-referencias.bib}
%\printbibliography

%\bibliographystyle{apalike}
%\bibliography{tcc-nome-referencias}


%\addbibresource{tcc-nome-referencias}

%\printbibliography

%\printindex

\end{document}