\centerline{\bfseries RESUMO}
\label{resumo}
\vspace{8mm}

Conteinerização é uma ferramenta muito útil quando se lida com projetos que precisam de diferentes dependências ou programas que podem ter conflitos entre si, que precisam de uma grande quantidade de configuração inicial, ou que precisam de portabilidade. Isso a torna perfeita para projetos de robótica, mas os impactos do seu uso e suas peculiaridades em situações reais ainda não estão documentadas, o que é justamente oque este projeto propõe fazer. Haverão 2 partes para este projeto, a primeira parte será uma avaliação do desempenho em uma simulação utilizando Gazebo, e a segunda parte será a avaliação do desempenho de um turtlebot real.  

Palavras-chave: Robótica, ROS, Docker, Conteinerização
