\centerline{\bfseries RESUMO}
\label{resumo}
\vspace{8mm}

Containerização é uma ferramenta muito útil quando se lida com projetos que precisam de diferentes dependências ou programas que podem ter conflitos entre si, que precisam de uma grande quantidade de configuração inicial, ou que precisam de portabilidade. Este projeto visa identificar e mostrar a diferença de desempenho entre robôs, que utilizam o Robot Operating System 2 (ROS 2), simulados com e sem conteinerização. Para verificar esta diferença, os testes foram realizados no ambiente simulado do software Gazebo Classic.
\ante{Os resultados mostram que o desempenho da simulação não é impactado pelo uso do Docker, que ocorre um pequeno aumento do uso de memória, porém sem variações significativas no uso do processador}
\novo{Os resultados mostram que o apesar do aumento do uso de memória e da uma variação significative do uso de processador (segundo o Teste T realizado), o desempenho na simulação não é impactado pelo uso do Docker}, indicando que a simulação usando containers é viável durante o processo de desenvolvimento.

\textbf{Palavras-chave:} Robô, Robot Operating System 2 (ROS 2), Gazebo Classic, Docker, Containerização.