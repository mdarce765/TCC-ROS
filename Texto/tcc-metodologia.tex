\chapter{METODOLOGIA}
\label{metodologia}
A metodologia (Figura \ref{fig:Fluxograma}) é dividida em 2 partes, sem Docker e com Docker, respectivamente, sendo que a integração com o Docker é na parte de navegação, isolando a stack nav2 e rviz do host. Todos os testes são executados em uma arena similar à presente na sala K4-04 da FEI. Foi utilizado o programa nmon para obter os dados durante os testes, obtendo o desempenho do sistema inteiro, servindo para garantir que todas as estatísticas estão sendo comparadas de forma íntegra. Foram utilizados scripts bash para tornar a execução dos testes consistente, tendo cada parte do teste também em scripts, facilitando a criação/modificação de testes personalizados. Foi criado um pacote ROS2 (tcc) para organizar scripts, mapas e arquivos de lançamento, sendo que alguns destes são modificações de arquivos presentes em outros pacotes deste projeto. Foi utilizado um arquivo modificado de AMCL (burger.yaml) para inicializar imediatamente a localização do TurtleBot, tornando os testes completamente autônomos e mais consistentes, já que a localização manual pode ser rejeitada. No primeiro teste, o robô simplesmente precisa percorrer a arena de ponta a ponta, tendo um SLAM pré-feito para o auxiliar, sem obstáculo algum. No segundo teste, o robô também precisa percorrer a arena de ponta a ponta com um SLAM pré-feito, mas dessa vez a arena está populada por outros robôs, que servem para atrapalhar o trajeto do robô principal, testando sua resiliência quanto à mudança de rotas com e sem Docker. 
Foram realizados testes em que o robô realizaria o mesmo objetivo dos dois primeiros testes, mas com o diferencial de a arena possuir um labirinto que pode ser montado na arena física (seguindo e respeitando as regras presentes na arena física da K4-04) como obstáculo para o robô. 
\cite{tccdockergithub} \footnote{As configurações e arquivos para facilitar o setup estão presentes no repositório \url{https://github.com/joca2511/TCC_Docker}}

\begin{figure}[htb]
    \centering
    \includegraphics[width=0.7\linewidth]{Figures/FluxogramaTCC.png}
    \caption{Fluxograma do projeto}
    \label{fig:Fluxograma}
\end{figure}