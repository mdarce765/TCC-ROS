\chapter{METODOLOGIA}
\label{metodologia}

\begin{figure}[htb]
    \centering
    \includegraphics[width=0.9\linewidth]{Figures/FluxogramaTCC.png}
    \caption{REFAZER PARA REMOVER ROBO REAL, TROCAR PARTE 2 PARA DOCKER/ROS, TIRAR BANCO DE DADOS}
    \label{fig:Fluxograma}
\end{figure}

A metodologia (Observada na \ref{fig:Fluxograma}) é dividida em 2 partes, sem Docker e com Docker , respectivamente, sendo que a integração com o Docker é na parte de navegação, isolando a stack nav2 e rviz do host. Todos os testes são executados em uma arena similar à presente na sala K404 da FEI.   
Foi utilizado o programa nmon para obter os dados durantes os testes, obtendo o desempenho do sistema inteiro, servindo para garantir que todas as estatistícas estão sendo comparadas de forma íntegra.
Foram utilizados scripts bash para tornar a execução dos testes consistente, tendo cada parte do teste também em scripts, facilitando a criação/modificação de testes personalizados.
Foi criado um pacote ROS2 (tcc) para organizar scripts, mapas e arquivos launch, sendo que alguns destes são modificações de arquivos presentes em outros pacotes deste projeto.
Foi utilizado um arquivo modificado de AMCL (burger.yaml) para imediatamente inicializar a localização do turtlebot, tornando os testes completamente autônomos e mais consistêntes, já que a localização manual pode ser rejeitada.
No primeiro teste, o robô simplesmente precisa percorrer a arena de ponta a ponta, tendo um SLAM pré-feito para o auxiliar, sem obstáculo algum.
No segundo teste, o robô também precisa percorrer a arena de ponta a ponta com um SLAM pré-feito, mas dessa vez a arena está populada por outros robôs "patrulheiros", que servem para atrapalhar o trajeto do robô principal, testando sua resiliência quanto a mudança de rotas com e sem Docker.

