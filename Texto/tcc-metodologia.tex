\chapter{METODOLOGIA}
\label{metodologia}

\begin{figure}[htb]
    \centering
    \includegraphics[width=0.9\linewidth]{Figures/FluxogramaTCC.png}
    \caption{REFAZER PARA REMOVER ROBO REAL, TROCAR PARTE 2 PARA DOCKER/ROS, TIRAR BANCO DE DADOS}
    \label{fig:Fluxograma}
\end{figure}

A metodologia (Figura \ref{fig:Fluxograma}) é dividida em 2 partes, sem Docker e com Docker, respectivamente, sendo que a integração com o Docker é na parte de navegação, isolando a stack nav2 e rviz do host. Todos os testes são executados em uma arena similar à presente na sala K4-04 da FEI. No primeiro teste, o robô simplesmente precisa percorrer a arena de ponta a ponta, tendo um SLAM pré-feito para o auxiliar, sem obstáculo algum. No segundo teste, o robô também precisa percorrer a arena de ponta a ponta com um SLAM pré-feito, mas dessa vez a arena está populada por outros robôs "patrulheiros", que servem para atrapalhar o trajeto do robô principal, testando sua resiliência quanto à mudança de rotas com e sem Docker.
