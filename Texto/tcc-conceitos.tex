\chapter{CONCEITOS FUNDAMENTAIS} 
\label{conceitos}

Este capítulo aborda os conceitos teóricos e as ferramentas utilizadas no desenvolvimento deste trabalho, com o intuito de descrever e relacionar estes conceitos, para haver entendimento nas etapas posteriores abordadas. O principal conceito a ser abordado é a \textbf{Containerização:} uma forma de virtualização feita para ser mais rápida e flexível que a emulação, é um processo de implantação que consegue ser executado em diversos dispositivos e sistemas operacionais, isso ocorre pelo fato de que um contêiner consegue armazenar os arquivos e bibliotecas para ser executado, permitindo a um usuário executar uma aplicação de outro sistema operacional no sistema operacional que o mesmo possua, além disso, o contêiner permite que falhas ocorram sem afetar outros processos que não estão agrupados no mesmo.\cite{Wen2023}. 
A Containerização será feita usando o \textbf{Docker:} É uma plataforma utilizada para desenvolvimento, envio e funcionamento de aplicações de maneira separada da infraestrutura por conta da conteinerização, por conta deste fator o usuário consegue gerir as aplicações da mesma maneira que gera sua infraestrutura, outro fator importante é que o Docker permite que as aplicações desenvolvidas sejam testadas e executadas com menos atraso do que a maneira convencional. Contêineres são bons para fluxos de integrações e entregas de trabalho contínuas.\cite{dck2025}
Para realizar os testes no robô simulado, será utilizado o \textbf{ROS2(Robot Operating System 2):} que é um meta-sistema operacional de código aberto utilizado para auxiliar a desenvolver aplicações para robôs, o mesmo possui serviços que outros sistemas operacionais normalmente possuem, mas com o foco maior para a área da robótica, facilitando comunicação entre processos, funções que se comunicam com as demais e entre muitos outros. Para o desenvolvimento do projeto, será utilizado o ROS 2 que mantém o conceito modular e distribuido, mas possui melhorias e mais funcionalidades que o ROS original (Figura \ref{fig:ROS25}) \cite{ros2025}

\textbf{Meta-sistema operacional:} Um meta-sistema operacional se trata de um sistema operacional para robôs, o mesmo realiza todas as funções que um sistema operacional faz. O mesmo possui bibliotecas e ferramentas que executam códigos em multiplos computadores. Este sistema operacional é utilizado em robôs...

\textbf{Middleware:} Uma camada de software que conecta as aplicações a um sistema operacional, permitindo uma comunicação e compartilhamento de dados mais simples entre os componentes de um sistema, esta facilidade permite que os desenvolvedores foquem no desenvolvimento das principais funções de uma aplicação, pois a comunicação entre a aplicação e o sistema operacional está sendo feita pelo middleware.

\textbf{DDS:} Um protocolo middleware e uma API para conexão centrada em dados, este protocolo integra os componentes de um sistema que muitas aplicações precisam, como arquitetura escalar, confiabilidade e prover conectividade de dados de baixa latência. Este protocolo foi criado pela OMG.

\textbf{Gazebo Simulator:} É um software usado para desenvolver simulações, possui diversos projetos de código aberto para que os interessados possam utilizar e desenvolver suas próprias simulações. Neste software estão presentes também diversos modelos, tanto como objetos como também robôs.(Figura \ref{fig:gzb25}) \cite{gzb2025}

\textbf{TurtleBot3 Burger:} É um robô customizável de preço acessível ao público baseado no modelo ROS para ser utilizado como um material educativo, de pesquisas, entretenimento pessoal e etc, é um robô que foi desenvolvido com o intuito de ser barato, por conta disto, o mesmo não possui uma grande funcionalidade ou qualidade, mas o mesmo compensa na relação da quantidade de aplicações que o mesmo consegue realizar.(Figura \ref{fig:tbt3b25}) \cite{turtlebot3_manual}

\textbf{Fast DDS:} Implementação de DDS feita em c++, possui uma biblioteca que oferece uma API e protocolo de comunicação que disponibiliza um modelo Publisher-Subscriber centrado em dados (DCPS). Este modelo visa ser eficiente e confiável para distribuir as informações para o sistema em tempo real.

\textbf{Cyclone DDS:} Implementação de DDS com alto desempenho, permite que os desenvolvedores que o utilizam possam criar "gêmeos" digitais das entidades de seus sistemas permitindo compartilhar estados, eventos, fluxos de dados e mensagens pela rede em tempo real, visa ser rápido, consistente e seguro. 

\begin{figure}[htb]
    \centering
    \includegraphics[width=0.5\linewidth]{Figures/ROS.png}
    \caption{TROCAR ESSA IMAGEM \cite{cop2020}}
    \label{fig:ROS25}
\end{figure}

\newpage
\begin{figure}[htb]
    \centering
    \includegraphics[width=0.5\linewidth]{Figures/gazebo.png}
    \caption{Exemplo utilizando o Gazebo Simulator \cite{gzb2025}}
    \label{fig:gzb25}
\end{figure}

\begin{figure}[htb]
    \centering
    \includegraphics[width=0.3\linewidth]{Figures/TurtleBot3.jpg}
    \caption{TurtleBot3 Burger \cite{tbt2025}}
    \label{fig:tbt3b25}
\end{figure}