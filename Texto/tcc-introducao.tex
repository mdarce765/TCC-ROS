\chapter{INTRODUÇÃO}
\label{intro}

A conteinerização é uma ferramenta poderosa no campo de desenvolvimento e implementação, disponibilizando uma camada de isolamento entre componentes de um projeto, assegurando que estes não irão conflitar, seja por funções internas ou dependências de versões diferentes utilizadas. Na robótica, conteinerização é vista como uma técnica para facilitar o desenvolvimento, portabilidade e consistência em projetos de robótica, mas há um problema com relação a isso, 
\textcolor{red}{não foram feitas pesquisas detalhando sobre a mudança de desempenho que ocorre entre a integração do ROS 2 com relação ao Docker,} \textcolor{blue}{existem poucas pesquisas sobre essa mudança de desempenho entre a integração do ROS 2 com o Docker}, 
esta integração pode levar a perdas de desempenho para o robô, causando atraso de mensagens recebidas ou travando o robô, esse atraso pode ser causado por conta da sobrecarga do sistema por conta dos recursos a mais utilizados pelo Docker.

\section{OBJETIVO}
O objetivo deste projeto é o de documentar e avaliar o desempenho de um robô simulado com e sem conteinerização em arenas simuladas baseadas na arena da sala K4-04 do Centro Universitário FEI sendo executado no software Gazebo Classic.
\section{ESTRUTURA DO TRABALHO}
O decorrer deste projeto está dividido da seguinte maneira: na seção 2, serão abordados os conceitos e ferramentas utilizados no projeto junto para que o leitor possa entender com clareza o tema abordado no projeto e os termos utilizados.

Na seção 3, são abordados os trabalhos que possuem alguma relação com o projeto, como métricas, ideias e etc.

A Seção 4, é abordada a metodologia utilizada para o desenvolvimento deste projeto, demonstrando as técnicas utilizadas e os passos a serem realizados para atingir o objetivo final.

Na seção 5, são abordados os experimentos e resultados obtidos ao longo do projeto, como os experimentos foram executados, os problemas obtidos e, no final, mostrar os resultados que foram obtidos, explicando o que era esperado e o que acabou por ser obtido. 

Na seção 6, são abordadas a discussão, conclusão e os trabalhos futuros. São explicados os resultados obtidos do projeto, o que foi concluído e, no final, explicar alguns passos que não foram realizados e que estão disponíveis para algum aluno desenvolver no futuro.
