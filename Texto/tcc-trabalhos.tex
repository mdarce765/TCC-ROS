\chapter{TRABALHOS RELACIONADOS} \label{trabs}

De acordo com \cite{Wen2023}, seu projeto auxilia com relação ao entendimento da conteinerização em sistemas embarcados e também com relação ao seu desempenho. No artigo, é detalhada a utilização de diferentes formas de conteinerização e seu efeito no desempenho em diferentes tipos de hardware. Foram realizados testes gerais que envolviam CPU, memória, rede e disco, em três ambientes diferentes: máquina virtual, contêiner e um contêiner dentro de uma máquina virtual. Com isso, foi concluído que as máquinas virtuais e os contêineres possuem um desempenho semelhante ao bare-metal (servidor físico que é de uso exclusivo para apenas um cliente, sem a camada de virtualização que fica entre o hardware e o sistema operacional), onde entre a CPU, rede e memória, sofria uma perda de no máximo 5\% enquanto no disco a diferença era de até 35\%. Foi observado que o Docker e a KVM (máquina virtual baseada no Kernel) foram 5 a 10\% mais lentos que o bare-metal, com o Docker sendo mais lento ainda na primeira inicialização, mas levando a concluir que conteinerização e virtualização podem ser utilizados em aplicações para automóveis.
\begin{comment}
    This paper presents various benchmark experiments of hypervisors and containers in the context of SDVs. Several virtualization and containerization technologies are evaluated, including KVM, Docker, Podman, and Nspawn. We consider the performance of bare-metal as the baseline. Our experiments are conducted on different platforms, including a Raspberry Pi, a standard desktop, and a high-performance workstation. In the first part of our work, we perform a general performance test with selected benchmark tools regarding CPU, memory, network, and disk. Afterwards, we evaluate the performance in an automotive scenario with the Autoware framework in a VM, a container, and a container inside of a VM. Furthermore, we separate a single Autoware framework into different containers managed by the container orchestrator k3s and analyze its performance. The benchmark results show that the performance of running software in virtualized/containerized environments is comparable to the performance of bare-metal systems. In general benchmark tests, software running in VMs and containers have approx. 0-5% decrease in the performance of CPU, memory, and network. Virtualization and containerization show a greater impact on disk performance. The container engines have approx. 5-15% performance loss on disk. The disk performance in KVM is 35% slower than that in a bare-metal setup. We further design automotive application-related experiments to measure the application’s startup time in different software execution environments. The results show that startup times for KVM and Docker (excluding the first run) are 5 to 10% slower than those for bare-metal. Additionally, k3s performs better than bare- metal when Autoware is divided into nine containers. These results illustrate that virtualization and containerization are appropriate for automotive applications. In future work, we plan to demonstrate the microservice- based architecture for a containerized Autoware framework and ROS applications and investigate the impacts on SDVs.
\end{comment}

Segundo o autor \cite{SMKD2024}, seu projeto auxilia no entendimento dos conceitos de avaliação do Docker com relação a outros sistemas operacionais. Para verificar a diferença entre os sistemas operacionais, foram utilizados os recém-instalados Sistemas Operacionais que eram MacOS Ventura, Ubuntu 22.04 e Windows 10 rodando em um MacBook Pro 13. Os testes consistiam em "estressar" o Docker com relação à CPU, rede e na resiliência do mesmo. Após realizar os testes, foi observado que o sistema operacional possui uma importante influência sobre o desempenho presente no container docker, alguns possuíam benefícios com relação a outros em uma determinada categoria, o macOS se destacou com relação nos dados obtidos nas configurações utilizadas nos sistemas docker, não sofrendo grandes perdas de desempenho, o Linux se mostrou mais eficiente no geral, por conta de ser uma funções que raramente utilizam de escrita no disco, se mostrando uma escolha melhor que o MacOS, o Windows acabou não se beneficiando tanto quanto os outros, a não ser pela taxa de tranferencia de rede entre os contêineres, assim como o linux o windows possui problemas com a velocidade em que a escrita é feita e com isso. Este artigo auxilia no entendimento com relação aos tipos de testes que podem ser realizados para analisar o desempenho do Docker, auxiliando de uma maneira que possa ser um pontapé inicial para o desenvolvimento dos testes com o Docker, e como medir o desempenho de um contêiner.

