\chapter{TRABALHOS RELACIONADOS} \label{trabs}

O projeto de Wen et. al. (2023), seu projeto auxilia com relação ao entendimento da conteinerização em sistemas embarcados e também com relação ao seu desempenho. No artigo, é detalhada a utilização de diferentes formas de conteinerização e seu efeito no desempenho em diferentes tipos de hardware. Foram realizados testes gerais que envolviam CPU, memória, rede e disco, em três ambientes diferentes: máquina virtual, contêiner e um contêiner dentro de uma máquina virtual. Com isso, foi concluído que as máquinas virtuais e os contêineres possuem um desempenho semelhante ao bare-metal (servidor físico que é de uso exclusivo para apenas um cliente, sem a camada de virtualização que fica entre o hardware e o sistema operacional), onde entre a CPU, rede e memória, sofria uma perda de no máximo 5\% enquanto no disco a diferença era de até 35\%. Foi observado que o Docker e a KVM (máquina virtual baseada no Kernel) foram 5 a 10\% mais lentos que o bare-metal, com o Docker sendo mais lento ainda na primeira inicialização, mas levando a concluir que conteinerização e virtualização podem ser utilizados em aplicações para automóveis. \cite{Wen2023}.

O trabalho de Sobieraj e Kotyński (2024), seu projeto auxilia no entendimento dos conceitos de avaliação do Docker com relação a outros sistemas operacionais. Para verificar a diferença entre os sistemas operacionais, foram utilizados Sistemas Operacionais recém-instalados que eram MacOS Ventura, Ubuntu 22.04 e Windows 10 rodando em um MacBook Pro 13. Os testes consistiam em estressar o Docker com relação à CPU, rede e na resiliência do mesmo. Após realizar os testes, foi observado que o sistema operacional possui uma importante influência sobre o desempenho presente no container docker, alguns possuíam benefícios com relação a outros em uma determinada categoria, o macOS se destacou com relação aos dados obtidos nas configurações utilizadas nos sistemas docker, não sofrendo grandes perdas de desempenho, se mostrando extremamente versátil,
o Linux se mostrou mais eficiente quanto às aplicações que raramente utilizam escrita no disco, se mostrando uma escolha melhor que o MacOS com relação a bancos de dados em memória, cache e entre outros, pois por não necessitarem de tanta escrita, essas aplicações se beneficiam mais quando executadas em um container com Linux, 
o Windows acabou não se beneficiando tanto quanto os outros, a não ser pela taxa de tranferencia de rede entre os contêineres, assim como o Linux o Windows possui problemas com a velocidade em que a escrita é feita e com isso. Este artigo auxilia no entendimento com relação aos tipos de testes que podem ser realizados para analisar o desempenho do Docker, auxiliando de uma maneira que possa ser um pontapé inicial para o desenvolvimento dos testes com o Docker, e como medir o desempenho de um contêiner. \cite{SMKD2024}.

\begin{comment}
     5.1. macOS Verdict
    From the data gathered on our configuration, macOS is considered the most versatile system for operating with Docker containers. It does not suffer from any significant bottlenecks besides the host-to-container network speed visualized in Figure 10. As value of this throughput—469 Mbps—cannot be considered low, with a high probability it would match most requirements on a nonproduction setup, making Mac OS a solid operating system for a container-driven workflow. It must be pointed out that this system is not a silver bullet, and there are cases where other systems will benefit more.

    5.2. Linux Verdict
    Despite the preference for macOS as the most versatile host system for Docker containers, it might not always be the optimal choice. In instances where a program rarely utilizes write-to-disk functionality, Linux may prove as a more efficient solution. In-memory databases, data streaming platforms, database cache, static websites, FTP read-only servers are among the systems that infrequently require write operation and can benefit from being run in a Docker container on a Linux host.
\end{comment}

Conforme Wen et. al. (2024), o artigo aborda a arquitetura baseada em microserviços para sistemas automotivos. Cada serviço foi realizado em contêineres separados, pois os testes realizados identificaram que este método é viável e acaba por melhorar a latência existente em sistemas operacionais Linux sem contêineres, que obteve uma latência de 5 a 8\% end-to-end, além de reduzir a latência máxima, o que mostra a vantagem no uso de conteinerização para os sistemas automotivos em tempo real. Foi concluído que o ROS 2, utilizado para avaliar a arquitetura de microserviços para uma aplicação real de direção autônoma, foi de extrema importância por conta de sua arquitetura distribuída que é baseada em nós e possui comunicação DDS, o que levou ao isolamento das funções do veículo e facilitou a migração para contêineres. Na conteinerização, o ROS 2 perde um pouco de seu desempenho, mas em aplicações complexas como o Autoware, a mesma acaba por melhorar, reduzindo o uso de CPU e memória. Este artigo auxilia no entendimento do uso da conteinerização com relação ao ROS 2, seus problemas e seus acertos. \cite{Wen2024}.
