\chapter{TRABALHOS RELACIONADOS} \label{trabs}

Para o desenvolvimento do projeto, foram relacionados alguns artigos que auxiliam no desenvolvimento do mesmo, O \cite{Wen2023} auxilia com relação ao entendimento da conteinerização em sistemas embarcados e também com relação ao seu desempenho. No artigo, é detalhada a utilização de diferentes formas de conteinerização e seu efeito no desempenho em diferentes tipos de hardware.

O \cite{SMKD2024} auxilia no entendimento dos conceitos de avaliação do Docker com relação a outros sistemas operacionais. Para verificar a diferença entre os sistemas operacionais, foram utilizados os recém-instalados Sistemas Operacionais que eram MacOS Ventura, Ubuntu 22.04 e Windows 10 rodando em um MacBook Pro 13. Os testes consistiam em "estressar" o Docker com relação à CPU, rede e na resiliência do mesmo. Após realizar os testes, foi observado que o sistema operacional possui uma importante influência sobre o desempenho presente no container docker, alguns possuíam benefícios com relação a outros em uma determinada categoria, o macOS se destacou com relação nos dados obtidos nas configurações utilizadas nos sistemas docker, não sofrendo grandes perdas de desempenho, o Linux se mostrou mais eficiente no geral, por conta de ser uma funções que raramente utilizam de escrita no disco, se mostrando uma escolha melhor que o MacOS, o Windows acabou não se beneficiando tanto quanto os outros, a não ser pela taxa de tranferencia de rede entre os contêineres, assim como o linux o windows possui problemas com a velocidade em que a escrita é feita e com isso. Este artigo auxilia no entendimento com relação aos tipos de testes que podem ser realizados para analisar o desempenho do Docker, auxiliando de uma maneira que possa ser um pontapé inicial para o desenvolvimento dos testes com o Docker, e como medir o desempenho de um contêiner.

\begin{comment}
    By examining all results of the test procedure, it is clear that the operating system has a great influence on the behavior of the Docker container in terms of performance. The examined configurations appear to influence distinct aspects of container performance, such as CPU speed, memory operations, and networking throughput.
    Across all operating systems tested, some benefited in a given category, while encountering serious performance loss in other categories. A structure can be as quick as its slowest component; therefore, it is extremely important to balance performance on each individual aspect of the system.
5.1 macOS Verdict
    From the data gathered on our configuration, macOS is considered the most versatile system for operating with Docker containers. It does not suffer from any significant bottlenecks besides the host-to-container network speed visualized in Figure 10. As value of this throughput—469 Mbps—cannot be considered low, with a high probability it would match most requirements on a nonproduction setup, making Mac OS a solid operating system for a container-driven workflow. It must be pointed out that this system is not a silver bullet, and there are cases where other systems will benefit more.
5.2. Linux Verdict
    Despite the preference for macOS as the most versatile host system for Docker containers, it might not always be the optimal choice. In instances where a program rarely utilizes write-to-disk functionality, Linux may prove as a more efficient solution. In-memory databases, data streaming platforms, database cache, static websites, FTP read-only servers are among the systems that infrequently require write operation and can benefit from being run in a Docker container on a Linux host.
5.3. Windows Verdict
    Windows system does not benefit from any particular test case except for intercontainer network throughput. This unique feature can be utilized in network-based applications, such as distributed systems. Similar to Linux, this setup’s major drawback is the write speed; therefore, containers with high disk usage are not recommended for this configuration. Containers with Windows as host system could be used in applications recommended for the Linux system.
5.4. Critical Examination of Testing Methods
    Due to the methodology of this test, there is a possibility that some configurations will exceed Mac OS, especially when adjusting the discovered bottlenecks. This approach did not explore any performance tuning in addition to the default installation guide provided by the official Docker documentation and the assignment of hardware resources. Finding the right testing platform to conduct fair experiments is a challenging task. It is important to note that using a nonofficial Linux distribution due to T2 chip limitation and installing a Windows machine through external software might affect the performance of tested configurations.
\end{comment}