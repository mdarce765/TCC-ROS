\chapter{CONCLUSÃO}
\label{conclusao}

A partir dos resultados obtidos por meio dos experimentos realizados, pode ser constatado que o uso do Docker gera um impacto no desempenho do robô simulado. Esse impacto pode ser medido e observado por meio do consumo do desempenho da CPU, memória RAM e tráfego de rede, obtidos por meio da ferramenta nmon, que mediu os testes realizados no simulador gazebo, onde o robô realizou trajetos com e sem o uso do Docker. Foi observado que a diferença no desempenho foi mínima, mostrando que o uso do Docker não compromete o funcionamento da simulação. Com isso, foi concluído que o Docker pode ser utilizado, mas pode ter alguns problemas, pois não foram realizados testes com tarefas críticas ou com pouco recurso disponível, para realizar testes simulados utilizando o ROS 2, o que acaba por gerar praticidade e portabilidade para o usuário que não precisa se preocupar com prejuízos significativos. 

\begin{comment}
    A partir dos resultados obtidos nos experimentos realizados, constatou-se que o uso do Docker realmente gera impacto no desempenho de um robô simulado. Esse impacto pôde ser medido por meio da análise dos consumos de CPU, memória RAM e tráfego de rede, monitorados de forma detalhada utilizando ferramentas como o nmon, além dos próprios utilitários embarcados no Docker. Apesar disso, observou-se que a degradação de desempenho é mínima e não compromete o funcionamento geral do sistema. Assim, conclui-se que o Docker pode ser empregado de forma segura e eficiente em aplicações robóticas baseadas em ROS/ROS2, oferecendo praticidade e portabilidade sem introduzir prejuízos significativos ao usuário.
\end{comment}
\section{TRABALHOS FUTUROS}
Devido a problemas com relação ao uso da sala K4-04 e seus equipamentos e à falta de tempo, os testes físicos não foram realizados, isso acabou por alterar o tema do projeto que envolvia a comparação do desempenho do robô simulado com e sem contêiner, com o robô físico com e sem contêiner. Foi decidido que os testes não seriam realizados mas que estariam disponíveis para serem usados como um trabalho futuro.

Como trabalhos futuros, seria a obtenção e comparação de dados de testes reais e das simulações feitas com as ferramentas criadas por este projeto, facilitando a obtenção de dados da parte simulada.

