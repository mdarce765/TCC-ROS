\chapter{DISCUSSÃO E CONCLUSÃO}
\label{conclusao}

\begin{figure}[htb]
    \centering
    \includegraphics[width=1\linewidth]{Figures/Graficos.png}
    \caption{Gráficos resultados dos testes.}
    \label{fig:Resultados}
\end{figure}
\emph{Questão a ser respondida pela Conclusão/Discussões Finais: O que tudo isso significa?}
Esta seção mostra como o trabalho avançou o estado atual do conhecimento dentro da área de pesquisa.

Você deve fornecer uma justificativa clara para o seu trabalho na conclusão.  Além disso, você pode sugerir experimentos futuros e apontar os que estão em andamento.

\section{TRABALHOS FUTUROS}
Devido a problemas envolvendo a burocracia da FEI com relação ao uso da sala K4-04 e seus equipamentos e à falta de tempo, os testes físicos não foram realizados, isso acabou por alterar o tema do projeto que envolvia a comparação do desempenho do robô simulado com e sem contêiner, com o robô físico com e sem contêiner. O professor orientador conversou com a equipe e então foi decidido que os testes não seriam realizados mas que estariam disponíveis para serem usados como um trabalho futuro para algum próximo aluno que possa se interessar pelo tema e decidir continuar o desenvolvimento dos testes.

Como trabalhos futuros, seria a obtenção e comparação de dados de testes reais e das simulações feitas com as ferramentas criadas por este projeto, facilitando a obtenção de dados da parte simulada.