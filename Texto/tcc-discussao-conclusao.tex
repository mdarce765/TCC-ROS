\chapter{CONCLUSÃO}
\label{conclusao}

A partir dos resultados obtidos por meio dos experimentos realizados, \textcolor{red}{pode ser constatado que o uso do Docker gera um impacto no desempenho do robô simulado.} \textcolor{green}{MAIS CERTEZA NA AFIRMAÇÃO} Esse impacto pode ser medido e observado por meio do consumo do desempenho da CPU, memória RAM e tráfego de rede, obtidos por meio da ferramenta Nmon, que mediu os testes realizados no simulador gazebo, onde o robô realizou trajetos com e sem o uso do Docker. Foi observado que a diferença no desempenho foi mínima, mostrando que o uso do Docker não compromete o funcionamento da simulação. Com isso, foi concluído que o Docker pode ser utilizado, mas pode ter alguns problemas, pois não foram realizados testes com tarefas críticas ou com pouco recurso disponível, para realizar testes simulados utilizando o ROS 2, o que acaba por gerar praticidade e portabilidade para o usuário que não precisa se preocupar com prejuízos significativos. 

\begin{comment}
    A partir dos resultados obtidos nos experimentos realizados, constatou-se que o uso do Docker realmente gera impacto no desempenho de um robô simulado. Esse impacto pôde ser medido por meio da análise dos consumos de CPU, memória RAM e tráfego de rede, monitorados de forma detalhada utilizando ferramentas como o nmon, além dos próprios utilitários embarcados no Docker. Apesar disso, observou-se que a degradação de desempenho é mínima e não compromete o funcionamento geral do sistema. Assim, conclui-se que o Docker pode ser empregado de forma segura e eficiente em aplicações robóticas baseadas em ROS/ROS2, oferecendo praticidade e portabilidade sem introduzir prejuízos significativos ao usuário.
\end{comment}
\section{TRABALHOS FUTUROS}
Devido a problemas com relação ao uso da sala K4-04 e seus equipamentos e à falta de tempo, os testes em um robô real não foram realizados, isso acabou por alterar o escopo do projeto, removendo o componente real das avaliações, sendo decidido que os testes e avaliações do robô real seriam realizados como um trabalho futuro.

Como trabalho futuro, ficaria a avaliação e estudo da integração em um robô real, utilizando as ferramentas, arquivos e metodologias criadas por este projeto para facilitar a obtenção de dados, como a Dockerfile customizada, os scripts para fácil utilização e organização de vários comandos.

\textcolor{red}{Onde esta o trabalho de IHC com trabalho futuro?
O texto está muito raso. Falta aprofundamento. Provavelmente o grupo fez mais do que essa descrito.
Minha sugestão é que o grupo realize os testes no ambiente real e apresentem futuramente um texto mais completo.
Banca ROS2 +Docker
- O TurtleSim foi utilizado?
- Slide 5 e 6 - Podem colocar as imagens dos trabalhos relacionados no texto.
- O que dos trabalhos relacionados foi utilizado no trabalho?
- Ficou claro na apresentação o que eram os patrulheiro 
- Imagens nos slides dos experimentos devem estar no texto
- Gostaria de mais detalhes dos testes de navegação. Não faz sentido ele se perder. O ambiente estava mapeado?
- A explicação na apresentação está melhoro que do texto}
