\chapter{EXPERIMENTOS}
\label{experimentos}

\novo{Este capítulo apresenta os experimentos realizados que utilizam os conceitos apresentados no capítulo anterior.}

\novo{\section{EXPERIMENTOS EM SIMULAÇÃO}}
Os experimentos usam um modelo digital da arena física presente no laboratório da sala K4-04 da instituição 
\novo{e usam um robô TurtleBot3 Burger simulado disponível no ROS e Gazebo}. O motivo de esta
%\ante{arena}
\novo{combinação} ter sido escolhida se deu por conta da comparação que
%\ante{futuramente} pode ser realizada entre os testes virtuais e reais.

Foram
%\ante{feitos cinco}
\novo{propostos sete} experimentos com resultados satisfatórios, e um que acabou não fornecendo os resultados devidos por conta de instabilidades.
\novo{
Todos os testes foram feitos em um computador com as seguintes especificações:
Linux Jammy 22.04 com um processador AMD Ryzen 5 1600X (12 Cores, 3.6GHz), 16 GB RAM DDR4.
Todas as ferramentas e configurações destas estão disponíveis em um playbook, no repositório.
A imagem Docker utilizada no projeto se baseia na imagem usada para ROS 2 Humble, presente no repositório oficial da robotis\footnote{https://github.com/ROBOTIS-GIT/turtlebot3/tree/main/docker/humble}, modificada para ter as configurações e arquivos necessários.
}

%\ante{O primeiro deles (Setup)teve testes iniciais para confirmar o comportamento do ROS 2 dentro de um container Docker.No segundo experimento (Base), usaram-se simulações utilizando o mapa base sem patrulheiros. O terceiro experimento (Patrulheiros) usou simulações utilizando o mapa base com patrulheiros. O quarto experimento (Base + Docker) usou simulações utilizando o mapa base sem patrulheiros, com a stack nav2 e rviz dentro de um container Docker. No quinto experimento (Patrulheiros + Docker), usaram-se simulações utilizando o mapa base com patrulheiros, com a stack nav2 e rviz dentro de um container Docker.}
Foram também realizados experimentos usando arenas com labirintos para testar a navegação do robô,
%\ante{mas a navegação acabou tendo problemas em experimentos com e sem Docker, dando a indicar que é um problema com o mapa, ou com as bibliotecas ROS 2 utilizadas, e não com a conteinerização.} 

\novo{Os sete experimentos propostos foram:
\begin{enumerate}
    \item Setup: testes iniciais para confirmar o comportamento do ROS 2 dentro de um contêiner Docker e a viabilidade do projeto proposto.
    \item Arena: navegação do robô de um lado a outro na arena da sala K4-04 sem nenhum obstáculo, conforme o caminho exibido na Figura~\ref{fig:AB}.
    \item Patrulheiros: usa o mesmo mapa do experimento anterior, porém 3 robôs TurtleBot3 Burger foram adicionados para realizar um percurso pré-definido (Figura~\ref{fig:AP}) com o objetivo de atrapalhar a movimentação do robô que faz a navegação na arena. Os robôs patrulheiros faziam um trajeto em linha reta onde saiam e retornavam aos seus pontos de origem.
    \item Arena com Docker: Este experimento é idêntico ao experimento Arena, porém o robô executa a parte do ROS de dentro de um contêiner Docker.
    \item Patrulheiros com Docker: Assim como o anterior, este experimento repete o que foi realizado no experimento Patrulheiros mas com o robô usando o ROS em um contêiner.
    \item Labirintos: dois mapas com paredes (Figuras~\ref{fig:AL1} e~\ref{fig:AL2}) foram criados para verificar se ocorre alguma alteração no desempenho do robô em relação a navegação. O uso destas paredes se devem ao fato da arena física presente na K4-04 e os mapas desenhados respeitam o espaço da arena, a quantidade de placas disponíveis e as possíveis posições que estas podem ser adicionadas na arena.
    \item Labirintos com Docker: assim com os anteriores, este experimento replica o que foi proposto no Labirintos, mas com o robô executando o ROS de dentro do contêiner.
\end{enumerate}
}
\begin{figure}[ht!]
\caption{Modelos de arena usadas para os experimentos simulados.}
\centering
\begin{subfigure}[t]{0.45\textwidth}
    \caption{Arena base com robô TurtleBot3 Burger. A linha verde representa o trajeto que deve ser realizado pelo robô.}
    \includegraphics[width=\textwidth]{Figures/Arena/ArenaBase.png}
    \label{fig:AB}
\end{subfigure}
\hfill
\begin{subfigure}[t]{0.45\textwidth}
    \caption{Arena Patrulheiros com 4 robôs TurtleBot3 Burger. As retas em vermelho são os trajetos realizados pelos patrulheiros, enquanto a curva em verde, representa o trajeto do robô.}
    \includegraphics[width=\textwidth]{Figures/Arena/ArenaPatrulheiros.png}
    \label{fig:AP}
\end{subfigure}
\begin{subfigure}[b]{0.45\textwidth}
    \caption{Labirinto 1. Labirinto feito adicionando paredes à arena da K4-04.}
    \includegraphics[width=\textwidth]{Figures/Arena/ArenaLabirinto1.png}
    \label{fig:AL1}
\end{subfigure}
\hfill
\begin{subfigure}[b]{0.45\textwidth}
    \caption{Labirinto 2. Segunda arena montada para testes com obstáculos.}
    \includegraphics[width=\textwidth]{Figures/Arena/ArenaLabirinto2.png}
    \label{fig:AL2}
\end{subfigure}
\caption*{Fonte: Autores}
\end{figure}

\novo{Para todos os experimentos as métricas de uso de CPU, RAM e rede foram medidas usando o nmon e armazenadas em arquivos para que os resultados fossem analisados após a execução de todos os testes. Cada experimento foi executado pelo menos 30 vezes para garantir que a amostragem dos resultados é estatisticamente relevante.}

\novo{\section{Experimentos no robô real}}

\novo{Os experimentos utilizaram somente um robô TurtleBot3 Burger que utilizava a Raspberry Pi 4 Model B Rev 1.5, com 4 núcleos, utilizando um processador Quad core Cortex-A72 (ARM v8) 64-bit SoC 1.8GHz com 2GB de memória RAM. O robô utilizava também Ubuntu 22.04.5LTS, Docker 28.4.0 (esta imagem foi baixada do repositório do Ubuntu) e ROS 2 Humble Hawksbill. Os testes foram realizados somente na arena base (sem obstáculos) na sala K4-04 (Figura \ref{fig:AB}). Estes testes consistiam em repetir o teste de navegação simulada, mas na arena física. Neste teste, o robô precisa sair de uma ponta e se locomover até a outra. O teste iniciava quando o Nmon era acionado por meio de um script armazenando os dados da CPU, memória RAM e tráfego de rede, e então, com outro script, o robô inicializava o xterm que inicializava o rviz. Com isso, o robô começava a se locomover. Quando o robô finalizava o trajeto, eram executados os scripts para encerrar o Nmon e o xterm, com isso o robô era reposicionado e depois o teste era repetido mais 29 vezes, este teste foi executado tanto sem quanto com o Docker. Foram executados no total 60 testes, divididos entre sem e com Docker. Após os dados serem obtidos, os mesmos eram analisados no software NMONVisualizer e então convertidos para “.csv” para serem utilizados em gráficos}.

\section{CONTEÚDO REPOSITÓRIO}
\novo{
O repositório GitHub\footnote{Disponível em https://github.com/joca2511/TCC\_Docker} criado para este projeto possui várias ferramentas que podem ser utilizadas para facilitar a reprodução dos testes executados.
O arquivo \texttt{README.md} contém um overview de como instalar o projeto e utilizar os scripts presentes no projeto.
Foi criado um playbook ansible (arquivo \texttt{playbook.yaml}) para facilitar a configuração e instalação do Docker, ROS 2 Humble, Gazebo Classic e suas dependências.
Foram também criados vários scripts para shell (arquivos com extensão \texttt{.sh}) para facilitar a reprodução dos testes, para garantir que os comandos corretos serão executados na sequência correta, sem necessidade de intervenção do usuário, onde os arquivos \texttt{inicioRapido*.sh} possuem os comandos utilizados para cada um dos testes feitos para este projeto.
Os arquivos com extensão \texttt{.sh} na pasta \texttt{/scripts} possuem funcionalidades genéricas criadas para os testes, como a inicialização do nmon (script \texttt{iniciarNmon.sh}), mover o robô ao inicializar rviz e mandar uma mensagem de goal (\texttt{moverMain.sh}), entre outras.
No pacote ROS 2 criado (localizado na pasta /tcc/tcc) possui o arquivo \texttt{turtlebot3\_absolute\_move\_Arena.py} que é uma versão modificada do arquivo \texttt{turtlebot3\_absolute\_move.py}, utilizado na movimentação dos robôs patrulheiros.
Esta modificação faz com que os robôs inicializados entrem em um loop infinito de movimentação entre 2 pontos especificados, lógica utilizada no script \texttt{rotasRobos.sh}.
Há também uma versão modificada de \texttt{multi\_robot\_Arena.launch.py} com as posições iniciais dos robôs patrulheiros e do arquivo \texttt{turtlebot3\_world.launch.py} modificado em \texttt{turtlebot3\_Arena.launch.py}, que permite carregar o robô em uma posição fixa em qualquer mapa, contanto que o nome seja especificado e o arquivo \texttt{.world} presente em /tcc/worlds.
}

%\ante{\section{TESTES ARENA} Foram feitos testes para adquirir os dados de desempenho das simulações sem Docker, utilizando a arena base sem patrulheiros (Figura \ref{fig:AB}). Estes testes consistiam em fazer o robô percorrer um trajeto de uma ponta até a outra da arena. Durante a execução, foi utilizado o software Nmon para verificar o desempenho de CPU, memória RAM e rede, e assim armazenar os dados. Os testes com a arena base + Docker, o robô realizou o mesmo trajeto que o teste anterior de seguir de uma ponta da arena até a outra. Os testes foram realizados enquanto o software Nmon verificava o desempenho do sistema.}

%\ante{\section{TESTES PATRULHEIROS}}
Neste teste foram posicionados robôs patrulheiros que percorriam um
%\ante{simples trajeto} 
\novo{trajeto de linha reta onde saiam e retornavam as seus pontos de origem (Figura \ref{fig:AP})}
para servir de obstáculo para o robô que deveria percorrer o mesmo trajeto de antes. Novamente foi utilizada a arena base, e sem o uso do Docker. Foram feitos testes para adquirir os dados de desempenho de simulações com Docker, utilizando a arena base com patrulheiros.
