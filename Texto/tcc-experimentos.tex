\chapter{EXPERIMENTOS E RESULTADOS[EXPLICAR FIGURAS]}
\label{experimentos}
\section{EXPERIMENTOS}
Os experimentos utilizam uma arena praticamente idêntica à arena presente na instituição, na sala K4-04 (Figura \ref{fig:gzb25}). O motivo de esta arena ter sido escolhida se deu por conta da comparação que poderia futuramente ser realizada entre os testes virtuais e reais o que infelizmente não ocorreu. O modelo da arena foi fornecido pelo professor da instituição, Fagner Pimentel, que forneceu o arquivo ".sdf" utilizado.

Experimentos feitos:
Experimento Setup: Testes iniciais para confirmar o comportamento de ROS2 dentro de um container Docker
Experimento Base: Simulações utilizando o mapa base sem patrulheiros
Experimento Patrulheiros: Simulações utilizando o mapa base com patrulheiros
Experimento Base+Docker: Simulações utilizando mapa base sem patrulheiros, com a stack nav2 e rviz dentro de um container Docker
Experimento Patrulheiros + Docker: Simulações utilizando o mapa base com patrulheiros, com a stack nav2 e rviz dentro de um container Docker.
\section{RESULTADOS SETUP}
Foram feitos testes 
%\cite{github_tcc_docker}
Para implementar a parte simulada proposta, foi utilizado o manual \cite{turtlebot3_manual} e pacotes \cite{turtlebot3_github} disponíveis pelo grupo ROBOTIS, tornando o desenvolvimento desta parte rápido. A utilização e desenvolvimento dos projetos ROS2 dentro do Docker foi facilitada pela utilização de workflows no GitHub, onde as imagens de teste foram automaticamente construídas e publicadas como pacotes no repositório, o que reduziu o tempo do processo de construir as imagens localmente, que demorava de 20+ minutos para no máximo 5 minutos, além de também as disponibilizar para outros usarem.
%Foram feitos testes  para implementar a parte simulada proposta, sendo utilizado o manual \cite{turtlebot3_manual} e pacotes \cite{turtlebot3_github} disponíveis pelo grupo ROBOTIS, tornando o desenvolvimento desta parte rápido. A utilização e desenvolvimento dos projetos ROS2 dentro do Docker foi facilitada pela utilização de workflows no GitHub, onde as imagens de teste foram automaticamente construídas e publicadas como pacotes no repositório, o que reduziu o tempo do processo de construir as imagens localmente, que demorava de 20+ minutos para no máximo 5 minutos, além de também as disponibilizar para outros usarem.

Houve certas dificuldades para integrar o ROS2 dentro do contêiner Docker com o ROS2 nativo, que lidaria com a simulação Gazebo. Foi descoberto que o middleware utilizado por padrão pelo ROS2 Humble, FastDDS, não interage de forma consistente com Docker. Ele era capaz de compartilhar os tópicos entre os ambientes, mas causava falha na publicação e recebimento de mensagens, não mostrando nenhuma.

Para solucionar isso, o FastDDS foi substituído por outro middleware disponível para ROS2 Humble, CycloneDDS, o qual foi utilizado especificamente no container Docker.

\begin{figure}[htb]
    \centering
    \includegraphics[width=1\linewidth]{Figures/TestePadraoFalha.png}
    \caption{Falha utilizando RMW padrão (FastDDS)}
    \label{fig:exp1}
\end{figure}
\begin{figure}[htb]
    \centering
    \includegraphics[width=1\linewidth]{Figures/TesteCycloneSucesso.png}
    \caption{Sucesso utilizando RMW CycloneDDS }
    \label{fig:exp2}
\end{figure}
\begin{figure}[htb]
    \centering
    \includegraphics[width=1\linewidth]{Figures/ComposeFinal.png}
    \caption{Compose proposto para testes}
    \label{fig:exp3}
\end{figure}
\begin{figure}[htb]
    \centering
    \includegraphics[width=1\linewidth]{Figures/SimulacaoGazeboLancada.png}
    \caption{Simulação Gazebo é lançada pelo host}
    \label{fig:exp4}
\end{figure}
\begin{figure}[htb]
    \centering
    \includegraphics[width=1\linewidth]{Figures/GazeboLancado.png}
    \caption{Aparência inicial da simulação Gazebo}
    \label{fig:exp5}
\end{figure}
\begin{figure}[htb]
    \centering
    \includegraphics[width=1\linewidth]{Figures/BurgerMovido.png}
    \caption{Turtlebot3 Burger é movido para dentro da casa}
    \label{fig:exp6}
\end{figure}
\begin{figure}[htb]
    \centering
    \includegraphics[width=1\linewidth]{Figures/DockerAdquireTopicos.png}
    \caption{ROS2 Humble dentro do Docker adquire os novos tópicos}
    \label{fig:exp7}
\end{figure}
\begin{figure}[htb]
    \centering
    \includegraphics[width=1\linewidth]{Figures/DockerBuildaProjetos.png}
    \caption{ROS2 Humble dentro do Docker builda projetos}
    \label{fig:exp8}
\end{figure}
\begin{figure}[htb]
    \centering
    \includegraphics[width=1\linewidth]{Figures/DockerRodaProjetoDrive.png}
    \caption{ROS2 Humble dentro do Docker roda projetos para movimentação do robô dentro da simulação Gazebo presente no host}
    \label{fig:exp9}
\end{figure}
\begin{figure}[htb]
    \centering
    \includegraphics[width=1\linewidth]{Figures/DockerMovimentaRobo1.png}
    \caption{Turtlebot3 Burger é movimentado pelo ROS2 Humble presente dentro do conteiner Docker}
    \label{fig:exp10}
\end{figure}
\begin{figure}[htb]
    \centering
    \includegraphics[width=1\linewidth]{Figures/DockerMovimentaRobo2.png}
    \caption{Turtlebot3 Burger é movimentado pelo ROS2 Humble presente dentro do conteiner Docker}
    \label{fig:exp11}
\end{figure}
\section{RESULTADOS BASE}
--A EDITAR--
Foram feitos testes para adquirir os dados de desempenho de simulações sem Docker, utilizando a arena base sem patrulheiros.
1- Rodar o comando no diretório do repositório (INSERIR COMANDO)
2 - Imagens durante a execução do teste
3- Mostrar dados via nmonvisualizer
\section{RESULTADOS PATRULHEIROS}
Foram feitos testes para adquirir os dados de desempenho de simulações sem Docker, utilizando a arena base com patrulheiros.
1- Rodar o comando no diretório do repositório (INSERIR COMANDO)
2 - Imagens durante a execução do teste.
3- Mostrar dados via nmonvisualizer
\section{RESULTADOS BASE + DOCKER}
Foram feitos testes para adquirir os dados de desempenho de simulações com Docker, utilizando a arena base sem patrulheiros.
1- Rodar o comando no diretório do repositório (INSERIR COMANDO)
2 - Imagens durante a execução do teste.
3- Mostrar dados via nmonvisualizer
\section{RESULTADOS PATRULHEIROS + DOCKER}
Foram feitos testes para adquirir os dados de desempenho de simulações com Docker, utilizando a arena base com patrulheiros.
1- Rodar o comando no diretório do repositório (INSERIR COMANDO)
2 - Imagens durante a execução do teste.
3- Mostrar dados via nmonvisualizer.