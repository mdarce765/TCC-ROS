\chapter{EXPERIMENTOS E RESULTADOS[EXPLICAR FIGURAS]}
\label{experimentos}
\section{DESENVOLVIMENTO DAS ARENAS}
Para serem realizados os testes de conteinerização, foram desenvolvidas arenas simuladas no software Gazebo Classic. As arenas foram desenvolvidas com base na arena presente na instituição, na sala K4-04 (COLOCAR IMAGEM DA ARENA REAL E DA ARENA SIMULADA SEM NADA). O motivo de esta arena ter sido utilizada se deu por conta da comparação que poderia ser realizada entre os testes virtuais e reais. Foram desenvolvidas quatro arenas, duas principais e duas alternativas, que são iguais às arenas principais, mas com a possibilidade de adicionar robôs patrulheiros que seriam obstáculos para o robô principal. O modelo da arena foi fornecido pelo professor da instituição, Fagner Pimentel, que forneceu um arquivo ".sdf" para ser utilizado, para então serem adicionadas placas como obstáculos (as placas seriam o principal obstáculo por conta do respeito para com as regras da arena real), gerando as arenas de teste.

\section{TESTES}
Foram feitos testes para implementar a parte simulada proposta, sendo utilizado o manual \cite{turtlebot3_manual} e pacotes \cite{turtlebot3_github} disponíveis pelo grupo ROBOTIS, tornando o desenvolvimento desta parte rápido. A utilização e desenvolvimento dos projetos ROS2 dentro do Docker foi facilitada pela utilização de workflows no GitHub, onde as imagens de teste foram automaticamente construídas e publicadas como pacotes no repositório, o que reduziu o tempo do processo de construir as imagens localmente, que demorava de 20+ minutos para no máximo 5 minutos, além de também as disponibilizar para outros usarem.
%Foram feitos testes \cite{github_tcc_docker} para implementar a parte simulada proposta, sendo utilizado o manual \cite{turtlebot3_manual} e pacotes \cite{turtlebot3_github} disponíveis pelo grupo ROBOTIS, tornando o desenvolvimento desta parte rápido. A utilização e desenvolvimento dos projetos ROS2 dentro do Docker foi facilitada pela utilização de workflows no GitHub, onde as imagens de teste foram automaticamente construídas e publicadas como pacotes no repositório, o que reduziu o tempo do processo de construir as imagens localmente, que demorava de 20+ minutos para no máximo 5 minutos, além de também as disponibilizar para outros usarem.

Houve certas dificuldades para integrar o ROS2 dentro do contêiner Docker com o ROS2 nativo, que lidaria com a simulação Gazebo. Foi descoberto que o middleware utilizado por padrão pelo ROS2 Humble, FastDDS, não interage de forma consistente com Docker. Ele era capaz de compartilhar os tópicos entre os ambientes, mas causava falha na publicação e recebimento de mensagens, não mostrando nenhuma.

Para solucionar isso, o FastDDS foi substituído por outro middleware disponível para ROS2 Humble, CycloneDDS, o qual foi utilizado especificamente no contêiner Docker.

\begin{figure}[htb]
    \centering
    \includegraphics[width=1\linewidth]{Figures/TestePadraoFalha.png}
    \caption{Falha utilizando RMW padrão (FastDDS)}
    \label{fig:enter-label}
\end{figure}
\begin{figure}[htb]
    \centering
    \includegraphics[width=1\linewidth]{Figures/TesteCycloneSucesso.png}
    \caption{Sucesso utilizando RMW CycloneDDS }
    \label{fig:enter-label}
\end{figure}
\begin{figure}[htb]
    \centering
    \includegraphics[width=1\linewidth]{Figures/ComposeFinal.png}
    \caption{Compose proposto para testes}
    \label{fig:enter-label}
\end{figure}
\begin{figure}[htb]
    \centering
    \includegraphics[width=1\linewidth]{Figures/SimulacaoGazeboLancada.png}
    \caption{Simulação Gazebo é lançada pelo host}
    \label{fig:enter-label}
\end{figure}
\begin{figure}[htb]
    \centering
    \includegraphics[width=1\linewidth]{Figures/GazeboLancado.png}
    \caption{Aparência inicial da simulação Gazebo}
    \label{fig:enter-label}
\end{figure}
\begin{figure}[htb]
    \centering
    \includegraphics[width=1\linewidth]{Figures/BurgerMovido.png}
    \caption{Turtlebot3 Burger é movido para dentro da casa}
    \label{fig:enter-label}
\end{figure}
\begin{figure}[htb]
    \centering
    \includegraphics[width=1\linewidth]{Figures/DockerAdquireTopicos.png}
    \caption{ROS2 Humble dentro do Docker adquire os novos tópicos}
    \label{fig:enter-label}
\end{figure}
\begin{figure}[htb]
    \centering
    \includegraphics[width=1\linewidth]{Figures/DockerBuildaProjetos.png}
    \caption{ROS2 Humble dentro do Docker builda projetos}
    \label{fig:enter-label}
\end{figure}
\begin{figure}[htb]
    \centering
    \includegraphics[width=1\linewidth]{Figures/DockerRodaProjetoDrive.png}
    \caption{ROS2 Humble dentro do Docker roda projetos para movimentação do robô dentro da simulação Gazebo presente no host}
    \label{fig:enter-label}
\end{figure}
\begin{figure}[htb]
    \centering
    \includegraphics[width=1\linewidth]{Figures/DockerMovimentaRobo1.png}
    \caption{Turtlebot3 Burger é movimentado pelo ROS2 Humble presente dentro do conteiner Docker}
    \label{fig:enter-label}
\end{figure}
\begin{figure}[htb]
    \centering
    \includegraphics[width=1\linewidth]{Figures/DockerMovimentaRobo2.png}
    \caption{Turtlebot3 Burger é movimentado pelo ROS2 Humble presente dentro do conteiner Docker}
    \label{fig:enter-label}
\end{figure}

