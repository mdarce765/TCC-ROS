\centerline{\bfseries ABSTRACT}
\label{abstract}
\vspace{8mm}

Containerization is a very useful tool when dealing with projects that require different dependencies or programs that may conflict with each other, that need a large amount of initial configuration, or that need portability. This project aims to identify and show the performance difference between robots, using Robot Operating System 2 (ROS 2), simulated with and without containerization. To verify this difference, tests will be performed in the simulated environment of the Gazebo Classic software. The results show that the simulation performance is not impacted by the use of Docker, that there is a small increase in memory usage, but without significant variations in processor usage, indicating that simulation using containers is viable during the development process.

\textbf{Keywords:} Robot, Robot Operating System 2 (ROS 2), Gazebo Classic, Docker, Containerization.